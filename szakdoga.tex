%% Based on a TeXnicCenter-Template by Tino Weinkauf.
%%%%%%%%%%%%%%%%%%%%%%%%%%%%%%%%%%%%%%%%%%%%%%%%%%%%%%%%%%%%%


%%%%%%%%%%%%%%%%%%%%%%%%%%%%%%%%%%%%%%%%%%%%%%%%%%%%%%%%%%%%%
%% HEADER
%%%%%%%%%%%%%%%%%%%%%%%%%%%%%%%%%%%%%%%%%%%%%%%%%%%%%%%%%%%%%
\documentclass[a4paper,oneside,10pt]{report}
% Alternative Options:
%	Paper Size: a4paper / a5paper / b5paper / letterpaper / legalpaper / executivepaper
% Duplex: oneside / twoside
% Base Font Size: 10pt / 11pt / 12pt

\usepackage{hegyhati}

\usepackage{color}
\newcommand{\todo}[1]{\textcolor{red}{#1}}

%%%%%%%%%%%%%%%%%%%%%%%%%%
\usepackage{t1enc}
\usepackage[utf8]{inputenc}
\usepackage{lmodern} 




%%%%%%%%%%%%%%%%%%%%%%%%%%%%%%%%%%%%%%%%%%%%%%%%%%%%%%%%%%%%%
%% DOCUMENT
%%%%%%%%%%%%%%%%%%%%%%%%%%%%%%%%%%%%%%%%%%%%%%%%%%%%%%%%%%%%%
\begin{document}



\begin{titlepage}
\begin{center}
\Large
Pannon Egyetem

\vspace{10mm}
Műszaki Informatikai Kar

\vspace{10mm}
Rendszer- és Számítástudományi Tanszék

\vspace{10mm}
Gazdaságinformatikus BSc

\vspace{40mm}
\huge
SZAKDOLGOZAT

\vspace{10mm}
\LARGE
Veszprémi tömegközlekedést támogató okostelefon alkalmazás 

\vspace{10mm}
\Large
Böröndi Evelin

\vspace{40mm}
Témavezető: Dr. Hegyháti Máté

\vspace{10mm}
2017
\normalsize
\end{center}
\end{titlepage}



\pagestyle{empty} %No headings for the first pages.



\newpage
\Large
\begin{center}
	\textbf{KÖSZÖNETNYILVÁNÍTÁS}
\end{center}
\normalsize
\noindent
...

\newpage
\Large
\begin{center}
	\textbf{TARTALMI ÖSSZEFOGLALÓ}
\end{center}
\normalsize
\noindent

A nagyvárosok bonyolult és szerteágazó tömegközlekedése összetett hálózatot alkotva húzódik végig a város különböző pontjait érintve.
A tömegközlekedéshez tartozó útvonalak, megállók és indulási idők sokaságán az tud igazán kiigazodni, aki hosszabb ideje használja már. 
A város által nyújtott közösségi közlekedés használatához szükség lehet egy olyan eszközre, amely segítségével az utazóközönség gyorsan és kényelmesen szerez információt, ezzel eljutva céljukhoz.
Ez a lehetőség főleg a turistáknak, a várost még nem ismerő személyeknek nyújthat nagy segítséget.

Veszprémben jelenleg a tömegközlekedést autobuszjáratok szolgáltatják.
Bár a megyeszékhely nem tartozik a legnagyobb városok közé Magyarországon, menetrendje ennek ellenére mégis szerteágazó.
Emellett a városban sok olyan ember megfordul, akik nem ismerik ki magukat Veszprémben, például turisták illetve egyetemisták.
Számukra nagy hátrányt jelent, hogy nincsenek tisztában a buszmegállók elhelyezkedéséről, de sok esetben a saját pozíciójukat sem ismerik.
A városban egyelőre nem található olyan szolgáltatás, ami eleget tenne annak, hogy segítse az utazóközönséget a tájékozódásban.
A jelenlegi megoldások túlságosan statikusak azok számára, akik újonnan látogatnak Veszprémbe.
Igény lenne egy olyan megvalósításra, ahol térkép alapján tudnának tájékozódni a megállókról.

Munkám során erre a hiányra próbál megoldást nyújtani az Androidos platformra készült alkalmazás.
Segítségével a felhasználók képesek megtervezni az utazásukat a tömegközlekedés bevonásával.

\textbf{Kulcsszavak:} tömegközlekedés, Android, Veszprém, útvonaltervezés

\newpage

\Large
\begin{center}
	\textbf{ABSTRACT}
\end{center}
\normalsize
\noindent
Angol tartalmi összefoglaló...

\textbf{Keywords:} public transport, Android, Veszprém, route planning
\tableofcontents
\newpage
\listoffigures
\newpage

%======================================================================


%% Title Page %%%%%%%%%%%%%%%%%%%%%%%%%%%%%%%%%%%%%%%%%%%%%%%
%% ==> Write your text here or include other files.

%% The simple version:
%\title{Címoldal}
%\author{Böröndi Evelin}
%\date{} %%If commented, the current date is used.
%\maketitle

%% Inhaltsverzeichnis %%%%%%%%%%%%%%%%%%%%%%%%%%%%%%%%%%%%%%%
%\tableofcontents %Table of contents
%\cleardoublepage %The first chapter should start on an odd page.

\pagestyle{plain} %Now display headings: headings / fancy / ...



%% Chapters %%%%%%%%%%%%%%%%%%%%%%%%%%%%%%%%%%%%%%%%%%%%%%%%%
%% ==> Write your text here or include other files.

%\input{intro} %You need a file 'intro.tex' for this.


%%%%%%%%%%%%%%%%%%%%%%%%%%%%%%%%%%%%%%%%%%%%%%%%%%%%%%%%%%%%%

%1. fejezet
\chapter{Bevezetés}
\label{bev}
%1. fejezet

A nagyvárosokban található tömegközlekedés bonyolult és szerteágazó módon húzódik végig a város különböző pontjait érintve. 
A tömegközlekedéshez tartozó útvonalak, megállók és indulási idők sokaságán az tud igazán kiigazodni, aki hosszabb ideje használja már. 
A városba „idegenként” érkezők, turisták számára szükséges lehet egy olyan eszköz, melynek segítségével eljutnak a céljukhoz, gyorsan és kényelmesen tudják használni a város nyújtotta közösségi közlekedést. 

Veszprémben jelenleg a közlekedés e fajtáját az autóbuszok szolgálják ki. 
Bár a megyeszékhely nem tartozik a legnagyobb városok közé Magyarországon, mégis szerteágazó menetrendet tudhat magáénak. 
Emellett a városban gyakran fordulnak meg egész évben turisták, egyetemisták, akik számára a legnagyobb hátrány, hogy nem ismerik a buszmegállókat, esetlegesen a saját helyzetüket sem. 
Egyelőre még nem található olyan szolgáltatás városunkban, ami eleget tenne annak, hogy segítse az utazóközönséget a tájékozódásban.
Jelenleg több, a veszprémi tömegközlekedést segítő megoldással is találkozhatunk. 
Ezen megvalósítások ugyanakkor túlságosan is statikusak egy újonnan a városba látogató számára. 
Egyik ilyen fennálló probléma az, hogy megtudja az utas egy adott megállóban milyen buszok állnak meg, végig kell néznie az egész menetrendet, továbbá nem szolgálnak vizuális visszajelzéssel, azaz ugyan ismeri a megálló nevét, azonban nem tudja pontosan meghatározni a város mely területén található. 
Ilyen helyzetekben igény lenne egy olyan megoldásra, hogy térkép alapján is tudjanak tájékozódni, viszont a jelenlegi megoldások közül egyik sem felel meg az előzőekben felállított igényeknek a kielégítésére.
Erre a fennálló problémára nyújt megoldást a szakdolgozatomban megvalósított Android alkalmazás, amely hordozható, megbízható és gyors formában tájékoztatja az utazni kívánókat. 

A 2. fejezetben bemutatom az útvonaltervezést, mint szolgáltatást, illetve a városban működő jelenlegi megoldásokat. 
A 3. fejezetben ismertetem a program szükséges funkcióit, és az azokkal szemben támasztott követelményeket.
A 4. fejezetben bemutatom az elkészült alkalmazást felhasználói szinten.
Az 5. fejezetben kitérek a fejlesztésre részletesebben, az alkalmazás egyes részegségeire és azok implementációjára. 
Végül a 6. fejezetben felvázolok pár elképzelést az alkalmazás továbbfejlesztéséhez.
\newpage


%2. fejezet
\chapter{Veszprém tömegközlekedése}
\label{tom}
%2. fejezet

A fejezetben ismertetésre kerül az útvonaltervezés, mint szolgáltatás. 
Sok weboldal és telefonos alkalmazás segítségével tudjuk az utunkat előre megtervezni, továbbá különböző egyedi funkciókat is igyekeznek fejleszteni, ezzel csalogatva magukhoz a felhasználókat. 
A fejezet első részében az útvonaltervezést fogom bemutatni pár népszerű alkalmazáson keresztül, amely sikeresen elégíti ki az utazni vágyók igényeit. 
Ezután áttekintést adok Veszprém tömegközlekedéséről, hogy átfogó képet adjak a város közlekedési helyzetéről. 
Végül pedig bemutatom a jelenleg is a piacon lévő megoldásokat, amik a városban való tájékozódást segítik. 


%2.1
\section{Az útvonaltervezés és a Google Maps}
\label{utvonalterv}

Útvonaltervezőnek nevezzük az olyan szoftvereket, amelyek két földrajzi pont között keresnek optimális útvonalat egy keresőmotor segítségével. 
Ezen motorok gyakran intermodális működésűek. 
Már az 1970-es évektől használják a támogatás ezen fajtáját. 
Akkoriban ez annyit jelentett, hogy egy terminálos felhasználói interfészen keresztül csatlakozott a hívóközponthoz, és onnan érdeklődték meg a tömegközlekedéssel kapcsolatos információkat. 
Miután elterjedt az a szokás az emberek között, hogy maguknak tervezték meg a nyaralásokat, és nem vették igénybe az utazási irodák ügynökeit, elkezdtek fejlődni az interneten elérhető útvonaltervezők. 

A tervezők ebbe a fajtájába tartozik az akkoriban Google Transitként ismert útvonaltervező, ami napjainkban a Google Maps térképes szolgáltatás része. 
Az alkalmazás interneten és mobil eszközökre is elérhető, rengeteg plusz funkcióval. 
A térképes adatbázisát különböző partnerek segítségével szerzi be, de sok helyen (például a fejlődő országokban) a közösség frissíti a térképes adatokat. 
Az útvonaltervező funkció kezdetben gyalogos és autós közlekedés tervezésére volt képes, azonban 2007-ben integrálták a tömegközlekedést is az útvonaltervezésbe. 
Magyarországon 2011 óta kizárólag Budapesten érhető el a funkció. 

\Picture{Google Maps útvonaltervezés}{2/googlemaps}{width=10cm}

Ahogy \aref{fig_2/googlemaps}.\ ábrán is látszik, tervezéskor beállíthatjuk a közlekedési formát, hogy éppen gyalogosan vagy tömegközlekedéssel szeretnénk igénybe venni. 
Gyalogos és autós közlekedéskor az elérhető járdákat és autóutakat veszi figyelembe az alkalmazás, majd az eredményt kirajzolja a térképre, esetlegesen több elérhető opció esetén a többit szaggatott vonallal jelöli. 
Tömegközlekedés esetén is több lehetőséget kínál fel, ezekből több szempont alapján tudjuk kiválasztani a számunkra optimálisat. 
Ilyen szempont lehet a legkevesebb átszállállás, illetve legkevesebb gyaloglás. 
A Google Maps továbbá indulási időket is rendel a járatokhoz, így képesek vagyunk tervezni mostani időpillanathoz, illetve ha később szeretnénk csak utazni, azt is beállíthatjuk. 
Mindemellett rendelkezik élő menetrendinformációval, amit a Budapesti Közlekedési Központ hivatalos oldaláról szerez be. 
Itt láthatjuk ha felújítás, útlezárás miatt nem közlekednek járatok, vagy más útvonalon járnak, emiatt más megállókat érintenek, ezt láthatjuk \aref{fig_2/elojaratinfo}.\ ábrán is. 

\Picture{Élő útvonalinformáció}{2/elojaratinfo}{width=6cm}


\section{Veszprém tömegközlekedése}
\label{veszpremtomeg}

Veszprém tömegközlekedését jelenleg a város méretéből és rendelkezésre álló infrastruktúrájából adódóan autóbuszok adják. 
A buszokat a Balaton Volán Zrt. biztosítja az 1960-as évek óta. 
Veszprém tömegközlekedésének gondolata azonban már 1884-ben megfogalmazódott Czollenstein Ferenc által, aki omnibuszokat indított Veszprém és Balatonalmádi között. 
A jelenlegi közlekedési forma 28 vonalat foglal magában, és a város belterületén kívül közlekedik a közigazgatásilag a városhoz tartozó településekhez is, úgymint  Szabadságpuszta, Jutaspuszta, Kádárta és Gyulafirátót, valamint Csatár. 
A tömegközlekedés több átalakításon is átesett azóta, amíg el nem elérte a mai napi formáját, amit \aref{fig_2/veszprembuszhalozat}.\ ábra mutat.
\Picture{Veszprém tömegközlekedési hálózata}{2/veszprembuszhalozat}{width=10cm}

A menetrend integrálása lehetséges a Google Maps rendszerébe, így az útvonaltervezés funkció Veszprém városában is használható lehetne. 
A busztársaság részéről, egy meghatározott formátumú adatbázis továbbítása szükséges a Google felé, mivel ez csak Budapesten valósult meg, így csak a fővárosban érhető el Magyarországon belül ez a szolgáltatás. 


\section{Jelenlegi megoldások}
\label{megoldasok}

Veszprémben a tömegközlekedés támogatására jelenleg is létezik több megoldás. 
A busztársaság is igyekszik minél kielégítőbb segítséget nyújtani az utazóközönségének, hiszen fontos számára, hogy minél többen vegyék igénybe a tömegközlekedést. 
Továbbá léteznek olyan harmadik fél által készült eszközök is, amelyek szintén hozzájárulnak az információszerzéshez. 
Ezen alkalmazásokat fejlesztőik nem profitszerzési céllal készítették el, hanem csupán önkéntes alapon, az emberek megsegítésének céljával.
Funkcionalitásuk ezáltal elmarad egy céges környezetben, nagyobb fejlesztőgárda által készített szolgáltatástól, melytől nyereséget várnak a tulajdonosok. 
Ezek közül mutatok be pár ismertebb példát, amik jelenleg elérhetőek a piacon. 



\section*{ÉNYKK}
\label{enykk}

Az ÉNYKK vagyis az Észak-nyugat-magyarországi Közlekedési Központ felel a tömegközlekedés üzemeltetéséért. 
Weboldalukon található dokumentum magába foglalja az összes buszjáratot, és az azokhoz tartozó megállókat és indulási időket. 
\Aref{fig_2/menetrend}.\ ábrán látható módon szerepel egy buszjárat a dokumentumban. 
Ez a fajta statikus megoldás általános segítséget nyújt az utazni vágyóknak, ha már rendelkeznek információval a városról, például a megállók elhelyezkedését illetően. 

%túl nagy es pixeles 
\Picture{Az 1-es buszhoz tartozó menetrend táblázat}{2/menetrend}{width=6cm}

A társaság igyekszik több segítséget nyújtani az utasoknak, emiatt új funkciókat készítettek a weboldalra az elmúlt időben. 
Létrehoztak egy térképes funkciót, ahol a járatokat kiválasztva az alkalmazás felrajzolja ezen buszoknak az útvonalát a térképre. 
Továbbá elkezdtek fejleszteni egy útvonaltervező funkciót is, viszont ezek jelenleg kezdetleges formában működnek csak. 
A térképes szolgáltatásnál kiválaszthajtuk, hogy milyen buszjáratokra vagyunk kiváncsiak, és a program kirajzolja azokat a térképre, ahogy \aref{fig_2/kirajzoltjarat}.\ ábra mutatja. 

\Picture{Az 1-es és a 4-es járat útvonala}{2/kirajzoltjarat}{width=10cm}

\section*{BamBusz}
\label{bambusz}

Online felületen elérhető segítség, célközönsége főleg az egyetemisták. 
Kedvezőbb megoldást nyújt, mint a busztársaság oldala abból a szempontból, hogy nem kell átböngésznünk az egész dokumentumot az indulási időkért, hanem beállíthatjuk az indulási és az érkezési megállónkat. 
Ezt követően az oldal kilistázza nekünk azokat a buszokat és a hozzájuk tartozó indulási időket, amikkel eljuthatunk a célunkhoz \aref{fig_2/bambusz}.\ ábrán látható módon. 
Hátránya hasonlóan a hivatalos oldalhoz, hogy ismernünk kell a megállókat, ahhoz hogy használni tudjuk. 

\Picture{A BamBusz webes felülete}{2/bambusz}{width=10cm}

\section*{Veszprémi buszmenetrend}
\label{veszprbuszmen}

Okostelefonra elérhető alkalmazás, ami letisztultan, egyszerűen és gyorsan jeleníti meg a buszjáratokat külön menüpontba szedve, ahogy \aref{fig_2/veszpremappmenu}.\ ábrán látható. 
Előnye, hogy akár útközben tudunk információt szerezni az autóbuszok közlekedési rendjéről. 
Továbbá elérhető egy éves menetrendi naptár a főoldalon, ami segítségével megállapíthatjuk hogy milyen rend szerint közlekednek a buszok adott napokon. 
Ugyanakkor az alkalmazás nem naprakész, a naptár a tavalyi évet reprezentálja, illetve ebből kifolyólag a buszjáratok menetrendje is az elmúlt évre érvényes indulásokat mutatja.

\todo{kepek egymas melle szerkesztese}
\Picture{Az alkalmazás menüje}{2/veszpremappmenu}{width=10cm}
\Picture{Az alkalmazás főoldala a menetrendi naptárral}{2/veszpremapp}{width=10cm}
\newpage


%3. fejezet
\chapter{Követelmények, technológiák}
\label{kov}
%3. fejezet


%3.1
\section{Követelmények}
\label{kovetelmeny}

Az alkalmazás elkészítése előtt felállított követelmények, amiket a fejlesztés során figyelembe kellett venni, hogy az utazóközönség számára optimális alkalmazást készüljön el. 


\begin{itemize}
	\item Alacsony erőforrás felhasználás:
	\\
	Az egyik legfontosabb követelmény, hogy az alkalmazás kevés erőforrás igénybevételével is megfelelően működjön. 
	Ehhez szükség van arra, hogy a telefonon egy lokális adatbázis üzemeljen. 
	Így az alkalmazás használatakor nem kell internetkapcsolatot biztosítani az adatok elérhetőek lesznek a telefon adatbázisából.	
	\item -	Megbízható:
	\\
	Létre kell hozni egy olyan webes felületet, ahol az adatbázis kezelhető, változások esetén pedig módosíthatóak az adatok. 
	Ebből kifolyólag az alkalmazás mindig naprakészen szolgálja az információt a felhasználóknak.
	
\end{itemize}


%3.2
\section{Funkcionális követelmények}
\label{funkckov}

Az alábbi követelmények tartalmazzák az alkalmazással, illetve a weboldallal szemben támasztott funkcionális elvárásokat.

%3.3
\section{Felhasznált technológiák}
\label{resz3_2}




%4. fejezet
\chapter{Felhasználói kézikönyv}
\label{felhaszn}
%4. fejezet

A fejezetben bemutatásra kerül a szakdolgozat keretében fejlesztett Android alkalmazás felhasználói szemszögből.
Kifejtésre kerül, milyen menüpontok találhatóak az applikációban, és ezek milyen funkcionalitással rendelkeznek.
Továbbá ismertetem az adatok menedzselésére szolgáló weboldal felületét és működését, amely az admin felhasználók munkáját teszi könnyebbé.

%4.1
\section{Android alkalmazás}
\label{androidapp}
Ahogy \aref{fig_4/icon}.\ ábrán is látszik, az applikáció ikonja egy Veszprém járműparkjában is szereplő autóbusz, az Ikarus 280.
\Picture{Az alkalmazás ikonja}{4/icon}{width=2cm}
Az alkalmazás indítása után az alkalmazás lekéri az adatbázisból azokat az adatokat, amik módosítási dátuma későbbi, mint az utolsó letöltés ideje.
Ezt a felhasználó egy felugró ablak képében látja, amelyen az alkalmazás közli, hogy frissítés van folyamatban, és kéri a felhasználók türelmét.
Ezt \aref{fig_4/fomenu}.\ ábra a) képén tekinthető meg.
Amint az adatok aktualizálása befejeződött, a felugró ablak eltűnik, átadva helyét a főmenü menüpontjainak, melyet \aref{fig_4/fomenu}.\ ábra b) része mutat.
\subsection{Főmenü}
\label{fomenu}
A főmenü az alkalmazás nevét és négy almenüt foglal magába:
\begin{itemize}
	\item Útvonaltervezés
	\item Menetrendek
	\item Megállók
	\item Kedvencek
\end{itemize}

Az applikáció témája a lila szín és annak árnyalatai, amely lehetővé teszi a felhasználók figyelmének felkeltését, mégis letisztult külsőt kölcsönöz.
\Picture{Főmenü}{4/fomenu}{width=10cm}

\subsection {Útvonaltervezés}
\label {utvonalterv}
Az Útvonaltervezés menüpontot kiválasztva egy Google Maps térkép jelenik meg Veszprém városára fókuszálva.
A képernyőn ezen kívül egy beviteli mező és három gomb található.
A beviteli mezőt a felhasználók a térképen való keresésre használhatják.
A keresés fő fókusza Veszprémre irányul, a mező automatikusan próbálja kiegészíteni a begépelt helyszínt, Veszprém környékére összpontosítva, ahogy az \aref{fig_4/autocomplete}.\ ábra a) képén is látható.
Az alkalmazás képes más földrajzi helyre irányuló kereséseket is kiegészíteni, viszont ez a funkció - az alkalmazás minél effektívebb működése érdekében - le van korlátozva.
\Picture{Helység keresése az applikációban}{4/autocomplete}{width=10cm}
Ahogy \aref{fig_4/autocomplete}.\ ábra b) képén is látszik, a találatot az alkalmazás egy úgynevezett 'marker' lerakásával jelöli a térképen.
A bal oldalon található gombbal a felhasználó képes a térkép nézetének megváltoztatására.
Az alapérmezett mód a domborzati megjelenés, a gomb megnyomásával átválthatunk műholdas nézetre.
A jobb oldalon két gomb helyezkedik el: a Saját pozíció, illetve az Útvonaltervezés.
A Saját pozíció gomb használatához szükség van GPS funkcióra a telefonban.
Ha ez nincs engedélyezve, az alkalmazás egy felugró ablak segítségével átirányítja a felhasználót a GPS funkció bekapcsolására alkalmas képernyőre.
\Picture{Útvonaltervezés saját pozícióval}{4/sajatpozUtvonalterv}{width=10cm}
Visszatérve az alkalmazásba, a térképen megjelenik a felhasználó feltételezhető helyzete egy kék ponttal jelölve.
Továbbá a térképen megjelenik egy világoskék kör az előbb említett kék pont körül.
Ezt \aref{fig_4/sajatpozUtvonalterv}.\ ábra a) képén láthatjuk.
A világosabb színezetű kör területén valamelyik pont a felhasználó biztos pozícióját jelöli, a kör nagysága az internet elérési módjától (mobilnet vagy Wifi) függ.
Ez a funkció azoknak a felhasználóknak nyújt segítséget, akik számára Veszprém városa ismeretlen terület.
A jobb alsó gombra kattintva a felhasználó átkerül az Útvonaltervezés oldalra, ahol a felső beviteli mező mellé egy újabb mező kerül, ahogy az \aref{fig_4/sajatpozUtvonalterv}.\ ábra b) részén is látszik. 
Ha az előző oldalon a mezőbe került már földrajzi hely, az alkalmazás automatikusan az alsó mezőt, az Utazás célját tölti fel vele.
Az új képernyőn megjelenő ablakban is lehetőség van a saját pozíció lekérésére, ebben az esetben az alkalmazás a paramétereket a Kiindulási pont mezőbe tölti be.
Abban az esetben, ha mind a kettő mezőt kitöltötte a felhasználó, akkor az Útvonaltervezés gomb ismételt megnyomásával átkerül a találati listára.
Az alkalmazás \aref{fig_4/uttervtalalat}.\ ábra a) részén látható módon jeleníti meg a találatokat.
\Picture{A találati képernyő és az térképes útvonalterv}{4/uttervtalalat}{width=10cm}
A képernyő felső részén a beállított indulási és érkezési helyszín van feltüntetve, alul pedig három féle szempont alapján a megoldás:
\begin{itemize}
	\item Legkevesebb utazási idő
	\\Ennél a lehetőségnél az alkalmazás kiszámolja a lehetséges útvonalakban a gyaloglással és utazással töltött idejét, és ezek közül a legrövidebbet jeleníti meg.
	\item Legkevesebb gyaloglás
	\\A második opció azt a találatot listázza ki, amelynél a gyaloglással töltött idő a legrövidebb.
	\item Legkevesebb átszállás
	\\Az utolsó alternatíva a jeggyel utazóknak kínál utazási megoldást, hiszen a legkevesebb átszállással járó utazást jeleníti meg.
\end{itemize}
A felhasználó az egyik megoldásra kattintva választhatja ki a számára optimális útvonalat.
Az alkalmazás a következő képernyőn az útvonalat térképen jeleníti meg, ahogy az \aref{fig_4/uttervtalalat}.\ ábra b) képén is látszik.
A kezdő és a végpont jelölője piros színű, a busz útvonalát az eltérő színnel (jelen esetben lilával) jelölt megállók mutatják.
Ha az útvonalban több busz is szerepel, akkor azok útvonalát egy harmadik színnel jelöli az alkalmazás.
A gyalogos útvonalat az applikáció színes vonallal jelöli a térképen a piros jelölők és az első/utolsó buszmegállók között.
Ha a választott megoldás mégsem felel meg a felhasználó igényeinek, a képernyő felső részén elhelyezkedő Útvonaltervezés gomb ismételt megnyomásával visszajut az utazási lehetőséget listázó oldalra.
\subsection {Menetrendek}
\label {menetrendek}
\Picture{Menetrendek}{4/menetrend}{width=10cm}
A Menetrendek a főmenü második menüpontja.
Ez az opció azokat a felhasználókat segíti, akik már ismerik Veszprém városát, illetve tömegközlekedését legalább alap szinten.
A menüpontra kattintva az alkalmazás kilistázza a buszjáratok számait, mely \aref{fig_4/menetrend}.\ ábra  a) képén tekinthető meg.
Amint kiválasztott a megtekinteni kívánt járatot, az applikáció bekéri a járat irányát is, majd kilistázza az indulási időket \aref{fig_4/menetrend}.\ ábra b) részén látható módon.
A megnyíló ablak fejléce tartalmazza az előbbi felhasználói akciók során bekért adatokat, illetve a Kedvencekhez adás funkciót egy szív ikon formájában.
A képernyő bal oldalán a járat megállói láthatóak, a választott iránytól függő sorrendben.
Az időpontok jobb oldalon, munkanapokra és szabadnapokra felosztva láthatóak.

\subsection {Megállók}
\label {megallok}
A Megállók menüpont alatt egy Google Maps térkép jelenik meg, rajta Veszprém város buszmegállóival.
A megállók pontos helyét piros színű jelölők jelzik a térképen, amely a közelítés mértékével arányosan csoportosítja a jelölőket.
A felhasználó megkeresi a térképen azt a megállót, amelyikre kíváncsi, az alkalmazás pedig kilistázza az ott megálló járatok számát, ahogy \aref{fig_4/megallokedvenc}.\ ábra a) képén is látszik.
\Picture{Megállók és Kedvencek}{4/megallokedvenc}{width=10cm}

\subsection {Kedvencek}
\label {kedvencek}
A menüpontban a Menetrendek opció alatt kedvencnek jelölt járatokat lehet elérni.
Az alkalmazás kilistázza azokat az adott menetirányhoz tartozó járatokat, amelyeket a felhasználó kedvencnek jelölt \aref{fig_4/megallokedvenc}.\ ábra b) ábráján látható módon.
A funkció egy szív ikon formájában jelenik minden menetrend adatlapján.
Alapértelmezetten a szív üres, rájuk kattintva tudja a felhasználó a Kedvencek menüpont alá rakni, ekkor a szív pirosra színeződik.
A listából eltávolítani hasonló módszerrel lehet, a telt szívre rákattintva kikerül a Kedvencekből.
A funkció gyors elérést biztosít így az adott járatok időpontjaihoz, ezzel meggyorsítva a tájékozódást.

%4.2
\section{Admin oldal}
\label{admin}
Az alkalmazás létrehozásakor szükség volt egy olyan kezelőfelület létrehozására is, amelyen keresztül könnyen kezelhetőek az adatbázisba feltöltött adatok.
Mivel az adatok közérdekűek és sok embert érintenek, fontos szempont volt, hogy csak azok férhessenek hozzá, akiknek van jogosultságuk.
\Picture{Autentikáció}{4/adminSignIn}{width=6cm}
Az oldal eléréséhez a felhasználónak igazolnia kell magát egy felhasználónév-jelszó párossal, ahogy az \aref{fig_4/adminSignIn}.\ ábrán is látszik.
\Aref{fig_4/adminStations}.\ ábrán látható admin oldal jelenik meg a sikeres bejelentkezés után.
Bal oldalon található a menü, amiből a felhasználó kiválaszthatja azt az elemet, amit menedzselni szeretne.
Alapértelmezetten a Járat menüpont jelenik meg, ez az adatbázis architektúrájának alapja.
A másik alapvető entitás a Megállók, amelyek a buszmegállók pontos földrajzi helyzetét tárolják.
\Picture{Adminisztrációs főoldal}{4/adminStations}{width=10cm}
Egy megálló felviteléhez hosszúsági és szélességi koordinátákra van szükség, amelynek felvitelét egy beágyazott Google Maps térkép segíti a weboldalon.
Ezt láthatjuk \aref{fig_4/adminStationAdd}.\ ábrán.
A felhasználó kijelöli a térképen a megálló pontos helyét, aminek a koordinátái megjelennek a térkép mellett és az adminisztrátornak már csak nevet kell hozzárendelnie az új rekordhoz.
Hasonló kezelőfelület található a megállók útvonalakhoz való rendelését kezelő menüpontban.
\Picture{Új megálló hozzáadása}{4/adminStationAdd}{width=10cm}
Egy új útvonal létrehozása esetén az adminisztrátor a Google Maps-en elhelyezkedő jelölők közül kiválasztja a megfelelőt.
Ezek a jelölők az adatbázisban található megállókat jelölik a térképen elhelyezve a könnyebb kezelés érdekében, ahogyan \aref{fig_4/adminLineHasStationAdd}.\ ábrán is látható.
Ha egy útvonal módosul (például egy felújítás miatt), akkor az egy új útvonalként felvezetésre kerül, az alapvető útvonal állapota pedig inaktívvá válik.
Amint az adminisztrátor sikeresen az adatbázishoz adott egy buszjáratot, a megfelelő útvonalhoz és járathoz indulási időket rendel.
Ezeket az időket előre definiált kategóriákba menthetjük el, ilyen kategória például a szabadnap, a munkanap vagy a tanszüneti munkanap.
Mindemellett előfordulhatnak olyan időpontok a naptárban, amikor nem az alapértelmezett közlekedési rend a mérvadó, például hétköznapra eső ünnepnapokon.
\Picture{Új útvonal hozzáadása}{4/adminLineHasStationAdd}{width=10cm}
Mivel ezek évről évre változnak, így az adminisztrátor feladata, hogy megfelelően felvezesse ezeket a napokat az adatbázisba.

Erre szolgál az Eltérő közlekedési rendek menüpont, ahol a megadott időintervallumra vonatkozóan felülírhatjuk az alapértelmezett rendet.

Az adatok átláthatóságának érdekében táblázatban jelennek meg az adatbázisból lekért rekordok.
Az adminisztrátor a keresés funkció segítségével kereshet az adatok attribútumai között, illetve sorba rendezheti azokat a gyorsabb kezelés érdekében.






















%5. fejezet
\chapter{Fejlesztői dokumentáció}
\label{fejl}
Az elkészült rendszer szerver-kliens architektúrára épült. 
Az architektúra modellje a \ref{fig_5/architektura}. ábrán látható. 
A szerver felelős az adatok kezeléséért, mely a Symfony keretrendszerrel lett megvalósítva. 
Az adatokat a Doctrine keretrendszer segítségével éri el az adatbázisból. 
Jelenleg kétféle kliens tartozik a rendszerhez, egy adminisztrációs feladatokat betöltő weboldal, és egy Android alkalmazás, amely a felhasználók részére készült. 
Az adminisztrációs weboldalon az adatok a Twig sablonkezelő segítségével jelennek meg. 
Az Android-os kliens a kért adatokat JSON formrmátumban kapja meg. 
A továbbiakban részletesen bemutatásra kerül a szerver és a kliensek felépítése, működése. 

\section{A szerver felépítése}
\label{szerverfelepites}

A szerver a Symfony keretrendszer 2.8-as verziójára épült. 
A beérkező kérés áthalad a Symfony tűzfalán, ami elindítja az autentikációt. 
Ha sikeres volt az autentikáció, a kérés a megfelelő kontroller osztályhoz kerül feldolgozás után. 

\Picture{Az autentikáció folyamata}{5/authenticationprocess}{width=14cm}
% tűzfal, autentikáció, ilyesmi...

A kontroller osztályban a beírt URL-hez tartozó metódus hívódik meg. 
Alapesetben ezek az osztályok a \Highlight{AppBundle\textbackslash Controller} névtérben találhatóak. 
Az URL-címek metódusokra való leképzése kétféleképp valósítható meg a Symfony keretrendszer használatával. 

\begin{itemize}
	\item routing.yml fájlban való definiálás
	\item kontroller osztályokban annotációként
\end{itemize}

Az első opció akkor előnyös, ha útvonal alapján akarjuk meghatározni, hogy mely kódrészlet fog végrehajtódni a kontrollerek megfelelő kialakításával és az annotációk jó elhelyezésével. 
Másrészről a második megoldás itt előnyösebb lehet, mert ilyenkor a kódot látva a fájl elhagyása nélkül meg tudjuk állapítani, hogy milyen esetekben fog a kódrészlet meghívásra kerülni. 
A metódusok annotációi között megszabhatunk különböző paramétereket is, melyet az alábbi kódrészlet demonstrál:

\begin{lstlisting}
    /**
    * @Route("/{id}/edit", name="line_edit")
    * @Method({"GET", "POST"})
    */
    public function editAction(Request $request, Line $line)
    {
        // ...
    }
\end{lstlisting}

\subsection*{Autentikáció}
\label{authentication}

Az autentikáció a Symfony tűzfal moduljával történik. 
Az ehhez szükséges beállítások a \Highlight{app\textbackslash config} mappa \Highlight{security.yml} konfigurációs fájljában találhatóak. 
A szerverhez jelenleg két fajta felhasználó van rendelve, ezek bele vannak égetve ebbe a fájlba. 
A szerver feladatköréből kiindulva a jelenlegi igényeket ez a két felhasználó kielégíti, mivel szükség újabb felhasználók hozzáadására. 
Ez azt jelenti, hogy a felhasználókhoz tartozó jelszavakat is ebben a fájlban tároljuk. 
Biztonsági szempontokból kritikus tényező, hogy ezek a jelszavak valamilyen módon titkosítva legyenek. 
A titkosításhoz a bcrypt algoritmust használtam fel. 
A Symfony segítségével parancssorból lehetséges bármilyen jelszót titkosítani a bcrypt algoritmussal. 

\begin{lstlisting}
php bin/console security:encode-password
\end{lstlisting}

A parancs futása közben bekéri, mi az a jelszó, amit titkosítani kell. 
Miután megadtuk a jelszót, a bemeneten lefuttatja az algoritmust és végül kiírja a titkosított karaktersorozatot. 
Ezt a szót megadva a fájlban jelszóként, továbbra is a titkosítatlan jelszóval be tud lépni a felhasználó az oldalra, viszont nem áll fent tovább a veszély, hogy illetéktelenek kezébe jutna a jelszó. 

\subsection*{Autorizáció}
\label{authorization}

Az autorizáció folyamata szintén a Symfony tűzfal moduljának segítségével valósult meg. 
Ez a sikeres bejelentkezést követően hajtódik végre. 
A folyamat célja, hogy a bejelentkezett felhasználó csak a számára kijelölt szolgáltatásokat, oldalakat érje el. 
Az előző pontban említett két felhasználóhoz tehát az alábbi két szerepkör tartozik:

\begin{itemize}
	\item Látogató
	\item Adminisztrátor
\end{itemize}

A látogató szerepkörhöz tartozó felhasználónak csak az adatbázis-entitások lekérdezéséhez van jogosultsága. 
Az adminisztrátori engedéllyel rendelkező viszont a weboldal minden szolgáltatásához hozzáfér. 

A szerepkörökhöz hozzá lettek rendelve meghatározott formájú URL címek. 

\begin{lstlisting}
access_control:
    - { path: /json$, roles: ROLE_USER }
    - { path: ^/, roles: ROLE_ADMIN }
\end{lstlisting}

Az alábbi kódrészlet azt mutatja be, hogy a felhasználói jogkör csak a \Highlight{\/json} végű URL-címekhez ad elérést, míg egy adminisztrátor minden címet elér. 
Ugyanezt a funkcionalitást meg lehet valósítani szintén annotációkkal is, de a weboldal kialakítása miatt célszerűbb volt ezt a megoldást választani, hisz így egyetlen fájlban elegendő meghatározni a hozzáférési szabályokat, annotációk használatával viszont minden kontroller osztályban egyenként kellett volna meghatározni ugyanazokat a szabályokat.

\subsection*{Adatbázis kapcsolat}
\label{adatbazis}

Az adatbázist a Doctrine keretrendszert használva éri el a szerver. 
Az alkalmazáshoz tartozó entitások helye a \Highlight{AppBundle\textbackslash Entity} névtér. 
Ezek mindegyike egyszerű PHP osztályok, a felhasznált keretrendszer alakítja ezeket az adatbázis tábláivá. 
Az osztályban szereplő adatmezőkhöz meg kell adni az annotációkat, hogy az átalakítás milyen módon történjen meg. 
Az entitások egymás közti kapcsolatait a következő annotációk jelzik:

\begin{itemize}
	\item OneToOne
	\item OneToMany
	\item ManyToOne
\end{itemize}

Az adatbázisban található adatok lekérdezés és objektumba való alakítása a \Highlight{Repository} osztályok segítségével történik. 
Ezek az osztályok a Doctrine \Highlight{EntityRepository} osztályából származnak, mely már rendelkezik a legalapvetőbb lekérdezésekkel, mint az azonosító vagy valamilyen tulajdonság alapján történő adatlekérés. 
Az összetettebb lekérdezésekhez a DQL nyelvet használtam fel, mely az entitások kapcsolatait az entitáshoz tartozó osztályban megadott annotációk alapján kezeli. 

Amennyiben az adatbázisból lekérdezett adatok módosultak, ahhoz, hogy ez a változás megmaradjon el kell menteni az adatbázisba. 
Szerencsére a Doctrine keretrendszernek van erre automatikus eszköze, mégpedig az EntityManager osztály, mely érzékeli a változást és elmenti a \Highlight{flush()} metódus használatával. 
Ha új adatot akarunk az adatbázishoz adni, azt is az EntitiyManager osztállyal tehetjük meg. 
Az elkészült új objektumot paraméterként átadva az osztály \Highlight{persist()} metódusának, a Doctrine keretrendszer úgy kezeli a továbbiakban az objektumot, mintha az az adatbázisból lekérdezett lenne, és így a \Highlight{flush()} metódussal az adatbázisba tudja írni. 

\section{Szerver feladatai}
\label{serverjob}

A alábbiakban bemutatásra kerülnek a szerver főbb feladatai.

\subsection*{Járatok}
\label{lineinfodetails}

A városban közlekedő buszok más-más végpontok között közlekednek. 
Előfordulhat, hogy ugyanolyan számmal ellátott busz ugyanazon végpontok között közlekedik, de a köztes megállók között eltérhet, vagy a köztes megállók ugyanazok, de a végpontok mások. 
Olyan helyzet is előfordulhat, hogy a két végpont megegyezik. 
Ezeknek az információk összefogására létrehozott \Highlight{LineInfio} entitás felelős. 
Ez az entitás tárolja a két végpontot, a közöttük közlekedő buszjáratot, és a járat típusát. 
A végpont és a buszjáratok szintén entitások, az ezek között fennálló kapcsolat az \ref{fig_5/lineinfo}. ábrán látható. 

\Picture{A járat entitás kapcsolatai}{5/lineinfo}{width=14cm}

\subsection*{Útvonalak}
\label{linehasstationsdetails}

A buszjáratok útvonaluk bejárása során több, köztes megállót is érintenek. 
Meg kell határoznunk, hogy egy adott busz milyen irányban haladva mely köztes megállókban áll meg. 
Ezt a feladatot a \Highlight{LineHasStations} entitás végzi. 
Az entitás segítségével meg tudjuk határozni, hogy az előző pontban vázolt járat egy példányának a végpontjain kívül milyen más állomásai vannak. 
Továbbá tároljuk a köztes megálló útvonalon belüli sorszámát, melyből megállapítható az útvonalon található állomások helyes sorrendje. 
Minden egyes objektum esetén tárolásra kerül még, hogy az előző megállóból milyen gyorsan tudunk eljutni. 
Természetesen a busz indulópontjának ezen attribútuma nulla. 

Az \ref{fig_5/linehasstations}. ábrán megtekinthető milyen kapcsolatokat tartalmaz az útvonal entitás. 

\Picture{A járathoz kapcsolatai}{5/linehasstations}{width=14cm}

\subsection*{Indulási idők, különleges időpontok}
\label{starttimes}

Útvonaltervezés szempontjából fontos információ továbbá, hogy az adott járat az útvonalán található megállóiba milyen időpontban érkezik meg, továbbá különleges alkalmak esetén ezek hogyan változnak meg. 
Az indulási időpontot egy Starttime nevű entitás tartalmazza, melyben tárolódik továbbá az is, hogy ez az indulás mely járatra vonatkozik, valamint milyen típusú dátum esetén értetendő. 
A különböző típusú dátumokat külön entitás tartalmazza. 
Az alkalmazáshoz tovább tartozik egy Holidays entitás is, melyben az olyan, előre nem tervezett időpontokat tartalmazza amelyek befolyásolják bizonyos járatok indulási rendjét és esetleges útvonalát is. 
Az előbbiekben bemutatott entitások közötti kapcsolatot a \ref{fig_5/starttimes}. ábra mutatja be. 

\Picture{Különleges és indulási időpontok közötti kapcsolat}{5/starttimes}{width=14cm}

%6. fejezet
\chapter{Továbbfejlesztési lehetőségek}
\label{tovabbfejl}
%6. fejezet

Az elkészült alkalmazás a felé támasztott követelményeknek eleget tesz, ennek ellenére funkcionalitása tovább bővíthető.
Fontos szempont, hogy a fejlesztés végeztével tesztelési céllal is ellenőrzésre kerüljün az applikáció.
Teszteléskor olyan hibák is jelentkezhetnek, amelyek a felhasználók által használt akciókból erednek.
Ennél a szakasznál lényeges lehet egy külső személy bevonása az ellenőrzésbe, ezáltal lecsökkentve az alkalmazás nem megfelelő használatából eredő hibák számát.
Mindemellett fontos, hogy az applikáció minél magasabb szinten elégítse ki a felhasználói igényeket, így továbbfejlesztésre és új funkciók bevezetésére is szükség lehet.

\subsection{Android Door framework}
\label{androiddoor}

\subsection{További városok hozzáadása}
\label{morecity}

\subsection{Útvonaltervezés továbbfejlesztése}
\label{routeplan}

\subsection{Mobil értesítések}
\label{pushnot}

\subsection{Widget}
\label{widget}












%7. fejezet
\chapter{Összefoglalás}
\label{osszefogl}
%7.fejezet

Szakdolgozatom célja egy olyan lehetséges megoldás elkészítése volt arra a problémára, hogy Veszprém szerteágazó tömegközlekedéséről mindenki gyorsan és kényelmesen tudjon informálódni.
Az alapötlet egy olyan program elkészítése volt, amely elérhető mindenki számára, és megbízható forrást nyújt az utazóközönségnek.

Mivel a város nem rendelkezik hasonló megoldással, ezért erre a hiányra alapozva készítettem el a \Highlight{Veszptrans} alkalmazást.
Fontos volt, hogy bárhol elérhető lehessen, emiatt Android operációs rendszerre készült el, amely a legelterjedtebb mobil operációs rendszer napjainkban. 
Több funkció is implementálva lett, figyelve arra, hogy a felhasználók több típusának igényeit is kielégítse.
Az útvonaltervezés opció mellett az alkalmazás tartalmazza a járatok menetrendjeit, de megtekinthető a megállók pontos helyzete a városban.
A jövőben könnyen illeszthetőek lesznek további funkcionalitások az alkalmazáshoz, mivel moduláris felépítésű.
Ilyen fejlesztés lehet például további városok hozzáadása, alkalmazás értesítések küldése, vagy egy asztali minimodul elkészítése.



%% <== End of hints
%%%%%%%%%%%%%%%%%%%%%%%%%%%%%%%%%%%%%%%%%%%%%%%%%%%%%%%%%%%%%


%%%%%%%%%%%%%%%%%%%%%%%%%%%%%%%%%%%%%%%%%%%%%%%%%%%%%%%%%%%%%
%% BIBLIOGRAPHY AND OTHER LISTS
%%%%%%%%%%%%%%%%%%%%%%%%%%%%%%%%%%%%%%%%%%%%%%%%%%%%%%%%%%%%%
%% A small distance to the other stuff in the table of contents (toc)
\addtocontents{toc}{\protect\vspace*{\baselineskip}}


\bibliographystyle{mybibstyle}
\bibliography{cite}

%% The List of Figures
%\clearpage
%\addcontentsline{toc}{chapter}{List of Figures}
%\listoffigures

%% The List of Tables
%\clearpage
%\addcontentsline{toc}{chapter}{List of Tables}
%\listoftables


\newpage

\Large
\begin{center}
	\textbf{MELLÉKLET}
\end{center}
\normalsize
\noindent
A mellékelt CD könyvtárszerkezete


% \begin{itemize}
%    \item \textbf{Dokumentum}
%    \begin{itemize}
%        \item \textbf{Forrás} - A szakdolgozat szerkeszthető formátumban
%        \item \textbf{Hivatkozások} - A szakdolgozatban lévő internetes hivatkozások letöltve
%        \item szakdolgozat.pdf
%    \end{itemize}
%    \item \textbf{Forrás} - A program forrásállománya
%    \begin{itemize}
%        \item \textbf{HexEngine}
%        \item \textbf{HexEngine-android}
%        \item \textbf{HexEngine-desktop}
%    \end{itemize}
%    \item \textbf{Program} - A futtatható program
%    \begin{itemize}
%        \item \textbf{bin} - A program grafikai és konfigurációs fájljai ami szükségesek az indításhoz
%        \item \textbf{hav} - A program a mentéseket tárolja itt
%        \item \textbf{java\_telepito} - A Java környezet telepítői
%        \item \textbf{libs} - Futtatáshoz kapcsolódó java fájlok
%        \item starter.jar
%    \end{itemize}
% \end{itemize}





%%%%%%%%%%%%%%%%%%%%%%%%%%%%%%%%%%%%%%%%%%%%%%%%%%%%%%%%%%%%%
%% APPENDICES
%%%%%%%%%%%%%%%%%%%%%%%%%%%%%%%%%%%%%%%%%%%%%%%%%%%%%%%%%%%%%
\appendix
%% ==> Write your text here or include other files.

%\input{FileName} %You need a file 'FileName.tex' for this.


\end{document}
