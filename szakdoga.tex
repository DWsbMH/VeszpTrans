%% Based on a TeXnicCenter-Template by Tino Weinkauf.
%%%%%%%%%%%%%%%%%%%%%%%%%%%%%%%%%%%%%%%%%%%%%%%%%%%%%%%%%%%%%


%%%%%%%%%%%%%%%%%%%%%%%%%%%%%%%%%%%%%%%%%%%%%%%%%%%%%%%%%%%%%
%% HEADER
%%%%%%%%%%%%%%%%%%%%%%%%%%%%%%%%%%%%%%%%%%%%%%%%%%%%%%%%%%%%%
\documentclass[a4paper,oneside,10pt]{report}
% Alternative Options:
%	Paper Size: a4paper / a5paper / b5paper / letterpaper / legalpaper / executivepaper
% Duplex: oneside / twoside
% Base Font Size: 10pt / 11pt / 12pt

\usepackage{hegyhati}


%%%%%%%%%%%%%%%%%%%%%%%%%%
\usepackage{t1enc} 
\usepackage{lmodern} 




%%%%%%%%%%%%%%%%%%%%%%%%%%%%%%%%%%%%%%%%%%%%%%%%%%%%%%%%%%%%%
%% DOCUMENT
%%%%%%%%%%%%%%%%%%%%%%%%%%%%%%%%%%%%%%%%%%%%%%%%%%%%%%%%%%%%%
\begin{document}



\begin{titlepage}
\begin{center}
\Large
Pannon Egyetem

\vspace{10mm}
Műszaki Informatikai Kar

\vspace{10mm}
Rendszer- és Számítástudományi Tanszék

\vspace{10mm}
Gazdaságinformatikus BSc

\vspace{40mm}
\huge
SZAKDOLGOZAT

\vspace{10mm}
\LARGE
Veszprémi tömegközlekedést támogató okostelefon alkalmazás 

\vspace{10mm}
\Large
Böröndi Evelin

\vspace{40mm}
Témavezető: Dr. Hegyháti Máté

\vspace{10mm}
2016
\normalsize
\end{center}
\end{titlepage}



\pagestyle{empty} %No headings for the first pages.



\newpage
\Large
\begin{center}
	\textbf{KÖSZÖNETNYILVÁNÍTÁS}
\end{center}
\normalsize
\noindent
...

\newpage
\Large
\begin{center}
	\textbf{TARTALMI ÖSSZEFOGLALÓ}
\end{center}
\normalsize
\noindent


Tartalmi összefoglaló...

\textbf{Kulcsszavak:} tömegközlekedés, Android, Veszprém, útvonaltervezés

\newpage

\Large
\begin{center}
	\textbf{ABSTRACT}
\end{center}
\normalsize
\noindent
Angol tartalmi összefoglaló...

\textbf{Keywords:} public transport, Android, Veszprém, route planning
\tableofcontents
\newpage
\listoffigures
\newpage

%======================================================================


%% Title Page %%%%%%%%%%%%%%%%%%%%%%%%%%%%%%%%%%%%%%%%%%%%%%%
%% ==> Write your text here or include other files.

%% The simple version:
%\title{Címoldal}
%\author{Böröndi Evelin}
%\date{} %%If commented, the current date is used.
%\maketitle

%% Inhaltsverzeichnis %%%%%%%%%%%%%%%%%%%%%%%%%%%%%%%%%%%%%%%
%\tableofcontents %Table of contents
%\cleardoublepage %The first chapter should start on an odd page.

\pagestyle{plain} %Now display headings: headings / fancy / ...



%% Chapters %%%%%%%%%%%%%%%%%%%%%%%%%%%%%%%%%%%%%%%%%%%%%%%%%
%% ==> Write your text here or include other files.

%\input{intro} %You need a file 'intro.tex' for this.


%%%%%%%%%%%%%%%%%%%%%%%%%%%%%%%%%%%%%%%%%%%%%%%%%%%%%%%%%%%%%

%1. fejezet
\chapter{Bevezetés}
\label{bev}
%1. fejezet

A nagyvárosokban található tömegközlekedés bonyolult és szerteágazó módon húzódik végig a város különböző pontjait érintve. 
A tömegközlekedéshez tartozó útvonalak, megállók és indulási idők sokaságán az tud igazán kiigazodni, aki hosszabb ideje használja már. 
A városba „idegenként” érkezők, turisták számára szükséges lehet egy olyan eszköz, melynek segítségével eljutnak a céljukhoz, gyorsan és kényelmesen tudják használni a város nyújtotta közösségi közlekedést. 

Veszprémben jelenleg a közlekedés e fajtáját az autóbuszok szolgálják ki. 
Bár a megyeszékhely nem tartozik a legnagyobb városok közé Magyarországon, mégis szerteágazó menetrendet tudhat magáénak. 
Emellett a városban gyakran fordulnak meg egész évben turisták, egyetemisták, akik számára a legnagyobb hátrány, hogy nem ismerik a buszmegállókat, esetlegesen a saját helyzetüket sem. 
Egyelőre még nem található olyan szolgáltatás városunkban, ami eleget tenne annak, hogy segítse az utazóközönséget a tájékozódásban.
Jelenleg több, a veszprémi tömegközlekedést segítő megoldással is találkozhatunk. 
Ezen megvalósítások ugyanakkor túlságosan is statikusak egy újonnan a városba látogató számára. 
Egyik ilyen fennálló probléma az, hogy megtudja az utas egy adott megállóban milyen buszok állnak meg, végig kell néznie az egész menetrendet, továbbá nem szolgálnak vizuális visszajelzéssel, azaz ugyan ismeri a megálló nevét, azonban nem tudja pontosan meghatározni a város mely területén található. 
Ilyen helyzetekben igény lenne egy olyan megoldásra, hogy térkép alapján is tudjanak tájékozódni, viszont a jelenlegi megoldások közül egyik sem felel meg az előzőekben felállított igényeknek a kielégítésére.
Erre a fennálló problémára nyújt megoldást a szakdolgozatomban megvalósított Android alkalmazás, amely hordozható, megbízható és gyors formában tájékoztatja az utazni kívánókat. % Do or don't there is no try

A 2. fejezetben bemutatom az útvonaltervezést, mint szolgáltatást, illetve a városban működő jelenlegi megoldásokat. 
A 3. fejezetben ismertetem a program szükséges funkcióit, és az azokkal szemben támasztott követelményeket.
A 4. fejezetben bemutatom az elkészült alkalmazást felhasználói szinten.
Az 5. fejezetben kitérek a fejlesztésre részletesebben, az alkalmazás egyes részegségeire és azok implementációjára. 
Végül a 6. fejezetben felvázolok pár elképzelést az alkalmazás továbbfejlesztéséhez.
\newpage


%2. fejezet
\chapter{Veszprém tömegközlekedése}
\label{tom}
%2. fejezet

A fejezetben ismertetésre kerül az útvonaltervezés, mint szolgáltatás. 
Sok weboldal és telefonos alkalmazás segítségével tudjuk az útunkat előre megtervezni, továbbá különböző egyedi funkciókat is igyekeznek fejleszteni, ezzel csalogatva magukhoz a felhasználókat. 
A fejezet első részében az útvonaltervezést fogom bemutatni pár népszerű alkalmazáson keresztül, amely sikeresen elégíti ki az utazni vágyók igényeit. 
Ezután áttekintést adok Veszprém tömegközlekedéséről, hogy átfogó képet adjak a város közlekedési helyzetéről. 
Végül pedig bemutatom a jelenleg is a piacon lévő segítséget, amik a városban való tájékozódást segítik. 


%2.1
\section{Az útvonaltervezés}
\label{utvonalterv}




\section{Veszprém tömegközlekedése}
\label{veszpremtomeg}



\section{Jelenlegi megoldások}
\label{megoldasok}




\section*{ÉNYKK}
\label{enykk}

Az ÉNYKK vagyis az Észak-nyugat-magyarországi Közlekedési Központ felel a tömegközlekedés üzemeltetése. 
Weboldalukon található dokumentum magába foglalja az összes buszjáratot, és az azokhoz tartozó megállókat és inulási időket. 
Ez a fajta statikus megoldás általános segítséget nyújt az utazni vágyóknak, ha már rendelkeznek információval a városról, például a megállók elhelyezkedést illetően. 
%pdfből kivágás járatról?
A társaság igyekszik több segítséget nyújtani az utasoknak, emiatt új funkciókat készítettek a weboldalra az elmúlt időben. 
Létrehoztak egy térképes funkciót, ahol a járatokat kiválasztva az alkalmazás felrajzolja ezen buszoknak az útvonalát a térképre. 
Továbbá elkezdtek fejleszteni egy útvonaltervező funkciót is, viszont ezek a jelenleg kezdetleges formában működnek csak. 
%térképes kirajzolásról kép



\section*{BamBusz}
\label{bambusz}

Online felületen elérhető segítség, célközönsége főleg az egyetemisták. 
Kedvezőbb megoldást nyújt, mint a busztársaság oldala abból a szempontból, hogy nem kell átböngésznünk az egész dokumentumot az indulási időkért, hanem beállíthatjuk az indulási és az érkezési megállónkat. 
Ezt követően az oldal kilistázza nekünk azokat a buszokat és a hozzájuk tartozó indulási időket, amikkel eljuthatunk a célunkhoz. 
Hátránya hasonlóan a hivatalos oldalhoz, hogy ismernünk kell a megállókat, ahhoz hogy használni tudjuk. 
%bambusz kép

\section*{Veszprémi buszmenetrend}
\label{veszprbuszmen}

Okostelefonra elérhető alkalmazás, ami letisztultan, egyszerűen és gyorsan jeleníti meg a buszjáratokat külön menüpontba szedve. 
Előnye, hogy akár útközben tudunk információt szerezni az autóbuszok közlekedési rendjéről. 
%kép az alkalmazásról??

\newpage









%3. fejezet
\chapter{Követelmények, technológiák}
\label{kov}
%3. fejezet

A fejezetben bemutatásra kerülnek a funkciókövetelmények, amelyeket a fejlesztés előtt és közben figyelembe kellett venni, hogy az utazóközönség számára egy mai igényeket kielégítő, modern alkalmazás készüljön el.  
Cél volt továbbá, hogy a rendszer másik szereplői, az adminisztrátorok számára is könnyen kezelhető, felhasználóbarát felület valósuljon meg.

%3.1
\section{Követelmények}
\label{kovetelmeny}

Az alábbi követelmények tartalmazzák azokat a tulajdonságokat, amiket az elkészült alkalmazásnak tartalmaznia kell.

\begin{itemize}
	\item Sebesség
	\\
	Válaszidő lecsökkentése minél alacsonyabb szintre. 
	A felhasználóknak fontos, hogy az alkalmazás segítségével gyorsan és megbízhatóan tudjanak információhoz jutni. 
	\item Alacsony erőforrás felhasználás
	\\
	Az egyik legfontosabb követelmény, hogy az alkalmazás kevés erőforrás igénybevételével is megfelelően működjön. 
	Ehhez szükség volt arra, hogy a telefon egy lokális adatbázist üzemeltessen. 
	Így az alkalmazás használatakor nem kell internetkapcsolatot biztosítani az adatok elérhetőek lesznek a telefon adatbázisából.	
	\item Megbízhatóság
	\\
	Létre kellett hozni egy olyan webes felületet, ahol az adatbázis kezelhető, változások esetén pedig módosíthatóak az adatok. 
	Ebből kifolyólag az alkalmazás mindig naprakészen szolgálja az információt a felhasználóknak.
	\item Könnyű kezelhetőség
	\\
	A felhasználók számára fontos, ha nincsenek bonyolult akciók, hanem lehetőleg az összes funkció használata egyértelmű, ezzel is gyorsítva az alkalmazás használatát. 
	Ha bizonyos jelölések, rövidítések igénylik, akkor súgót kell hozzáadni az alkalmazáshoz.
	
\end{itemize}


%3.2
\section{Funkcionális követelmények}
\label{funkckov}

Ebben a alfejezetben a felhasználóval és az adminisztrátorral kapcsolatos használati esetek kerülnek kifejtésre.

\section*{Android alkalmazás}
\begin{itemize}
	\item \textbf{Útvonaltervezés}
	\\
	Az útvonaltervezés az alkalmazás fő funkciója, ezért nagy hangsúlyt kellett a megvalósítására fektetni. 
	Könnyű kezelőfelületet kívánt, egyértelmű jelöléseket, amik bárki számára egyszerűvé teszik a használatát. 
	Ez a menüpont azokat az utasokat célozza meg elsődlegesen, akik számára Veszprém ismeretlen terület. 
	Fontos volt, hogy különböző utazási módozatok is elérhetőek legyenek, például azoknak, akik buszjegyet váltanak, minél kevesebb átszállással kínálja az alkalmazás az útvonalat. 
	Továbbá, mivel ez egy térképet integráló funkció, ezért megfelelő módon kellett megjeleníteni a térképet. 	
	\item \textbf{Menetrend}
	\\
	A menetrend funkció akkor kap szerepet egy felhasználónál elsősorban, amikor nincs elérhető internetkapcsolat a mobilkészülékén. 
	Ezen kívül azoknál, akik rendelkeznek a tömegközlekedésről legalább alapszintű ismerettel, így jártasak a menetrendben és az indulási időkről szeretnének informálódni. 
	Megjelenésénél figyelni kellett arra, hogy felülete letisztult, könnyen kezelhető és informatív legyen.
	\item \textbf{Megállók}
	\\
	Az alkalmazásban szükség volt egy olyan szolgálatásra is, ahol a felhasználók tájékozódni tudnak a városban a megállókról. 
	Mivel előfordulhat olyan eset, hogy valaki tisztában van a megálló elhelyezkedésével, viszont meg szeretné tudni, hogy milyen járatok érintik anélkül hogy az egész menetrendet át kéne olvasnia, ezért fontos volt implementálni egy ilyen funkciót az alkalmazásba. 
\end{itemize}
\section*{Weboldal}
\begin{itemize}
	\item Buszjáratok kezelése
	\item Menetidők kezelése
	\item Megállók kezelése
	\item Ideiglenes járatmodosulások kezelése
\end{itemize}
	
	

%3.3
\section{Felhasznált technológiák}
\label{felhaszntech}
\section*{Térkép}
Egy útvonaltervező alkalmazás nagyon fontos tulajdonsága a megjelenés. 
Egy ilyen alkalmazásnál alapvető elvárás, hogy a térképen megjelenő információ könnyen kiolvasható legyen, hogy a felhasználók könnyedén el tudják választani a lényeges információt azoktól, amelyek nem szükségesek a tájékozódáshoz. 
Emiatt szükségem volt egy olyan térképes szolgáltatásra, ami az előbbi céloknak megfelel. 
Mielőtt megkezdtem a fejlesztést, több ilyen szolgáltatást is megvizsgáltam, abból a célból, hogy a legmegfelelőbbet tudjam kiválasztani a tapasztalatok alapján a szakdolgozatomhoz. 
Az egyik legfontosabb szempont a kiválasztásnál az volt, hogy ingyenesen elérhető legyen a szolgáltatás. 
A kutatás során két fő jelöltre sikerült leszűkítenem a listát, a Google Maps-re illetve az OpenStreetMap-re. 
Mivel az általam készített programot Android platformra terveztem elkészíteni, ezért olyan térképes API-ra volt szükségem, amit be lehet építeni Android applikációba, és ennek a célnak mind a kettő szolgáltatás megfelelt. 
Mindezek mellett fontos volt számomra, hogy olyan térkép alkalmazást válasszak, ami felhasználói körökben jól ismert. 
Emiatt megvizsgáltam, hogy az alkalmazásoknak mekkora a felhasználói köre. 
Az OpenStreetMap Android alkalmazása körülbelül 5 millió  felhasználóval rendelkezik, míg a Google Maps letöltése meghaladja az 1 milliárd felhasználót. 
Ezekből az adatokból következtetni tudtam arra, hogy a felhasználók szélesebb köre miatt a Google Maps felülete sokkal szélesebb körben ismert az emberek között.

\Picture{A Google Maps és az OSM összehasonlítása}{3/terkeposszehas}{width=12cm}

\Aref{fig_3/terkeposszehas}.\ ábrán látható szempontok figyelembe vételével végül a Google Maps-re esett a választásom, mint integrálható térképes szolgáltatás.

\section*{Android, Java}
Az Android platformon\cite{androidprogramming} futó alkalmazások elsődleges programozási nyelve a Java.
A Java platformfüggetlen és széles körben elterjedt, a Sun Microsystems által fejlesztett nyelvet az 1990-es évek óta folyamatosan fejlesztik.
Az Android egy Linux alapú operációs rendszer, amely több pozitívummal is rendelkezik, ilyen például a hordozhatóság és a biztonság.
Futtatási sebessége a Java-hoz képes javult, hiszen a kibővített Java programozási nyelvet Android esetén egy olyan virtuális gépen futtatják, amely szerves részét képezi az Android operációs rendszernek. 
\todo{telenyomni hivatkozásokkal a részt}

\section*{Symfony}
A Symfony egy PHP alapú keretrendszer.
A vele készült webes alkalmazások könnyen karbantarthatóak és jól skálázhatóak, mivel a keretrendszer az MVC tervezési mintára épül.
A Symfony felépítése moduláris, ennek köszönhetően könnyen bővíthető.
Továbbá a Symfony egy-egy komponensét más keretrendszerrel készült alkalmazásban külön is használhatjuk, amit szintén a moduláris szerkezet tesz lehetővé.
Minden beérkező kérést a keretrendszer dolgoz fel, melyek az útvonalválasztás után meghívják a megfelelő kontrollerek metódusát.
Továbbá a kontrollerek hozzáférnek a services.yml fájlban definiált szolgáltatásokhoz is.
Az itt definiált szolgáltatások példányait, a Symfony dependency injection segítségével juttatja el a kontrollerekhez.
A Symfony segítségével a dinamikus tartalom is könnyen átalakítható a benne található template nyelvek egyikének köszönhetően.
Az általam választott nyelv a twig, amellyel a megjelenítés elválasztható az üzleti logikától. 

\section*{MySQL}
A MySQL\cite{mysql} egy SQL (Structured Query Language, vagyis struktúrált lekérdező nyelv) alapú relációs adatbázis-kezelő szerver.
Teljesen nyílt forráskodú, feltételezhetően emiatt terjedt el széles körben.
Egyszerűen használható és költséghatékony megoldást nyújt dinamikus webhelyek számára.
Az adatbázis kezelésére a phpMyAdmin adminisztrációs eszközt használtam, amely segítségével Interneten keresztül menedzselhettem az adatokat.

\section*{Doctrine}
A Doctrine keretrendszer az adatbázisban található adatok elérésére alkalmas.
A segítségével PHP osztályok képezhetőek le relációs adatbázis táblákká.
A keretrendszer lehetőséget ad arra, hogy az alkalmazás adatbázisa különösebb ráfordítás nélkül lecserélhető legyen.
A DQL, vagyis a Doctrine lekérdező nyelv használatával lehet az adatokat elérni.
A Doctrine véd továbbá az SQL injektálásos támadásokkal szemben is.
A Doctrine úgynevezett Repository-kat használ az adatok lekérdezésére, amelyek lehetővé teszik, hogy alkalmazás logikája és a lekérdezés megvalósítása elvonatkoztatható legyen egymástól.
Továbbá azért esett a választásom a Doctrine-ra, mert a keretrendszer integrálva van a Symfony-ba, illetve szolgáltatásként elérhető a Doctrine EntityManager objektuma.
Új entitások generálásában és az adatbázis tábláinak frissítésében is segítséget nyújt a Symfony parancssoros eszköze.















%4. fejezet
\chapter{Felhasználói kézikönyv}
\label{felhaszn}
%4. fejezet

A fejezetben bemutatásra kerül a szakdolgozat keretében fejlesztett Android alkalmazás felhasználói szemszögből.
Kifejtésre kerül, milyen menüpontok találhatóak az applikációban, és ezek milyen funkcionalitással rendelkeznek.
Továbbá ismertetem az adatok menedzselésére szolgáló weboldal felületét és működését, amely az admin felhasználók munkáját teszi könnyebbé.

%4.1
\section{Android alkalmazás}
\label{androidapp}
Ahogy \aref{fig_4/icon}.\ ábrán is látszik, az applikáció ikonja egy Veszprém járműparkjában is szereplő autóbusz, az Ikarus 280.
\Picture{Az alkalmazás ikonja}{4/icon}{width=2cm}
Az alkalmazás indítása után az alkalmazás lekéri az adatbázisból azokat az adatokat, amik módosítási dátuma későbbi, mint az utolsó letöltés ideje.
Ezt a felhasználó egy felugró ablak képében látja, amelyen az alkalmazás közli, hogy frissítés van folyamatban, és kéri a felhasználók türelmét.
Ezt \aref{fig_4/fomenu}.\ ábra a) képén tekinthető meg.
Amint az adatok aktualizálása befejeződött, a felugró ablak eltűnik, átadva helyét a főmenü menüpontjainak, melyet a \aref{fig_4/fomenu}.\ ábra b) része mutat.
\subsection{Főmenü}
\label{fomenu}
A főmenü az alkalmazás nevét és négy almenüt foglal magába:
\begin{itemize}
	\item Útvonaltervezés
	\item Menetrendek
	\item Megállók
	\item Kedvencek
\end{itemize}

Az applikáció témája a lila szín és annak árnyalatai, amely lehetővé teszi a felhasználók figyelmének felkeltését, mégis letisztult külsőt kölcsönöz.
\Picture{Főmenü}{4/fomenu}{width=10cm}

\subsection {Útvonaltervezés}
\label {utvonalterv}
Az Útvonaltervezés menüpontot kiválasztva egy Google Maps térkép jelenik meg Veszprém városára fókuszálva.
A képernyőn ezen kívül egy beviteli mező és három gomb található.
A beviteli mezőt a felhasználók a térképen való keresésre használhatják.
A keresés fő fókusza Veszprémre irányul, a mező automatikusan próbálja kiegészíteni a begépelt helyszínt, Veszprém környékére összpontosítva, ahogy az a \aref{fig_4/autocomplete}.\ ábra a) képén is látható.
Az alkalmazás képes más földrajzi helyre irányuló kereséseket is kiegészíteni, viszont ez a funkció - az alkalmazás minél effektívebb működése érdekében - le van korlátozva.
\Picture{Helység keresése az applikációban}{4/autocomplete}{width=10cm}
Ahogy a \aref{fig_4/autocomplete}.\ ábra b) képén is látszik, a találatot az alkalmazás egy úgynevezett 'marker' lerakásával jelöli a térképen.
A bal oldalon található gombbal a felhasználó képes a térkép nézetének megváltoztatására.
Az alapérmezett mód a domborzati megjelenés, a gomb megnyomávásával átválthatunk műholdas nézetre.
A jobb oldalon két gomb helyezkedik el: a Saját pozíció, illetve az Útvonaltervezés.
A Saját pozíció gomb használhatához szükség van GPS funkcióra a telefonban.
Ha ez nincs engedélyezve, az alkalmazás egy felugró ablak segítségével átirányítja a felhasználót a GPS funkció bekapcsolására alkalmas képernyőre.
\Picture{Útvonaltervezés saját pozícióval}{4/sajatpozUtvonalterv}{width=10cm}
Visszatérve az alkalmazásba, a térképen megjelenik a felhasználó feltételezhető helyzete egy kék ponttal jelölve.
Továbbá a térképen megjelenik egy világoskék kör az előbb említett kék pont körül.
Ezt \aref{fig_4/sajatpozUtvonalterv}.\ ábra a) képén láthatjuk.
A világosabb színezetű kör területén valamelyik pont a felhasználó biztos pozícióját jelöli, a kör nagysága az internet elérési módjától (mobilnet vagy Wifi) függ.
Ez a funkció azoknak a felhasználóknak nyújt segítséget, akik számára Veszprém városa ismeretlen terület.
A jobb alsó gombra kattintva a felhasználó átkerül az Útvonaltervezés oldalra, ahol a felső beviteli mező mellé egy újabb mező kerül, ahogy az \aref{fig_4/sajatpozUtvonalterv}.\ ábra b) részén is látszik. 
Ha az előző oldalon a mezőbe került már földrajzi hely, az alkalmazás automatikusan az alsó mezőt, az Utazás célját tölti fel vele.
Az új képernyőn megjelenő ablakban is lehetőség van a saját pozíció lekérésére, ebben az esetben az alkalmazás a paramétereket a Kiindulási pont mezőbe tölti be.
Abban az esetben, ha mind a kettő mezőt kitöltötte a felhasználó, akkor az Útvonaltervezés gomb ismételt megnyomásával átkerül a találati listára.
Az alkalmazás \aref{fig_4/uttervtalalat}.\ ábra a) részén látható módon jeleníti meg a találatokat.
\Picture{A találati képernyő és az térképes útvonalterv}{4/uttervtalalat}{width=10cm}
A képernyő felső részén a beállított indulási és érkezési helyszín van feltüntetve, alul pedig három féle szempont alapján a megoldás:
\begin{itemize}
	\item Legkevesebb utazási idő
	\\Ennél a lehetőségnél az alkalmazás kiszámolja a lehetséges útvonalakban a gyaloglással és utazással töltött idejét, és ezek közül a legrövidebbet jeleníti meg.
	\item Legkevesebb gyaloglás
	\\A második opció azt a találatot listázza ki, amelynél a gyaloglással töltött idő a legrövidebb.
	\item Legkevesebb átszállás
	\\Az utolsó alternatíva a jeggyel utazóknak kínál utazási megoldást, hiszen a legkevesebb átszállással járó utazást jeleníti meg.
\end{itemize}
A felhasználó az egyik megoldásra kattintva választhatja ki a számára optimális útvonalat.
Az alkalmazás a következő képernyőn az útvonalat térképen jeleníti meg, ahogy az \aref{fig_4/uttervtalalat}.\ ábra b) képén is látszik.
A kezdő és a végpont jelölője piros színű, a busz útvonalát az eltérő színnel (jelen esetben lilával) jelölt megállók mutatják.
Ha az útvonalban több busz is szerepel, akkor azok útvonalát egy harmadik színnel jelöli az alkalmazás.
A gyalogos útvonalat az applikáció színes vonallal jelöli a térképen a piros jelölők és az első/utolsó buszmegállók között.
Ha a választott megoldás mégsem felel meg a felhasználó igényeinek, a képernyő felső részén elhelyezkedő Útvonaltervezés gomb ismételt megnyomásával visszajut az utazási lehetőséget listázó oldalra.
\subsection {Menetrendek}
\label {menetrendek}
\Picture{Menetrendek}{4/menetrend}{width=10cm}
A Menetrendek a főmenü második menüpontja.
Ez az opció azokat a felhasználókat segíti, akik már ismerik Veszprém városát, illetve tömegközlekedését legalább alap szinten.
A menüpontra kattintva az alkalmazás kilistázza a buszjáratok számait, mely \aref{fig_4/menetrend}.\ ábra  a) képén tekinthető meg.
Amint kiválasztott a megtekinteni kívánt járatot, az applikáció bekéri a járat irányát is, majd kilistázza az indulási időket \aref{fig_4/menetrend}.\ ábra b) részén látható módon.
Az megnyíló ablak fejléce tartalmazza az előbbi felhasználói akciók során bekért adatokat, illetve a Kedvencekhez adás funkciót egy szív ikon formájában.
A képernyő bal oldalán a járat megállói láthatóak, a választott iránytól függő sorrendben.
Az időpontok jobb oldalon, munkanapokra és szabadnapokra felosztva láthatóak.

\subsection {Megállók}
\label {megallok}
A Megállók menüpont alatt egy Google Maps térkép jelenik meg rajta Veszprém város buszmegállóival.
A megállók pontos helyét piros színű jelölők jelzik a térképen, amely a közelítés mértékével arányosan csoportosítja a jelölőket.
A felhasználó megkeresi a térképen azt a megállót amelyikre kíváncsi, az alkalmazás pedig kilistázza az ott megálló járatok számát, ahogy \aref{fig_4/megallokedvenc}.\ ábra a) képén is látszik.
\Picture{Megállók és Kedvencek}{4/megallokedvenc}{width=10cm}

\subsection {Kedvencek}
\label {kedvencek}
A menüpontban a Menetrendek alatt Kedvencnek jelölt járatokat lehet elérni.
Az alkalmazás kilistázza azokat az adott menetirányhoz tartozó járatokat, amelyeket a felhasználó kedvencnek jelölt \aref{fig_4/megallokedvenc}.\ ábra b) ábráján látható módon.
A funkció egy szív ikon formájában jelenik minden menetrend adatlapján.
Alapérmezetten a szív üres, rájuk kattintva tudja a felhasználó a Kedvencek menüpont alá rakni, ekkor a szív pirosra színeződik.
A listából eltávolítani hasonló módszerrel lehet, a telt szívre rákattintva kikerül a Kedvencekből.

%4.2
\section{Admin oldal}
\label{admin}
Az alkalmazás létrehozásakor szükség volt egy olyan kezelőfelület létrehozására is, amelyen keresztül könnyen kezelhetőek az adatbázisba feltöltött adatok.
Mivel az adatok közérdekűek és sok embert érintenek, fontos szempont volt, hogy csak azok férhessenek hozzá akiknek van jogosultságuk.
\Picture{Autentikáció}{4/adminSignIn}{width=6cm}
Az oldal eléréséhez a felhasználónak igazolnia kell magát egy felhasználónév-jelszó párossal, ahogy az \aref{fig_4/adminSignIn}.\ ábrán is látszik.
\Aref{fig_4/adminStations}.\ ábrán látható admin oldal jelenik meg a sikeres bejelentkezés után.
Bal oldalon található a menü, amiből a felhasználó kiválaszthatja azt az elemet, amit menedzselni szeretne.
Alapértelmezetten a Járat menüpont jelenik meg, ez az adatbázis architektúrájának alapja.
A másik alapvető entitás a Megállók, amelyek a buszmegállók pontos földrajzi helyzetét tárolják.
\Picture{Adminisztrációs főoldal}{4/adminStations}{width=10cm}
Egy megálló felviteléhez hosszúsági és szélességi koordinátákra van szükség, amelynek felvitelét egy beágyazott Google Maps térkép segíti a weboldalon.
Ezt láthatjuk \aref{fig_4/adminStationAdd}.\ ábrán.
A felhasználó kijelöli a térképen a megálló pontos helyét, aminek a koordinátái megjelennek a térkép mellett és az adminisztrátornak már csak nevet kell hozzárendelnie az új rekordhoz.
Hasonló kezelőfelület található a megállók útvonalakhoz való rendelését kezelő menüpontban.
\Picture{Új megálló hozzáadása}{4/adminStationAdd}{width=10cm}
Egy új útvonal létrehozása esetén az adminisztrátor a Google Maps-en elhelyezkedő jelölők közül kiválasztja a megfelelőt.
Ezek a jelölők az adatbázisban található megállókat jelölik a térképen elhelyezve a könnyebb kezelés érdekében, ahogyan \aref{fig_4/adminLineHasStationAdd}.\ ábrán is látható.
Ha egy útvonal módosul (például egy felújítás miatt), akkor az egy új útvonalként felvezetésre kerül, az alapvető útvonal állapota pedig inaktívvá válik.
Amint az adminisztrátor sikeresen az adatbázishoz adott egy buszjáratot, a megfelelő útvonalhoz és járathoz indulási időket rendel.
Ezeket az időket előre definiált kategóriákba menthetjük el, ilyen kategória például a szabadnap, a munkanap vagy a tanszüneti munkanap.
Mindemellett előfordulhatnak olyan időpontok a naptárban, amikor nem az alapértelmezett közlekedési rend a mérvadó, például hétköznapra eső ünnepnapokon.
\Picture{Új útvonal hozzáadása}{4/adminLineHasStationAdd}{width=10cm}
Mivel ezek évről évre változnak, így az adminisztrátor feladata, hogy megfelelően felvezesse ezeket a napokat az adatbázisba.

Erre szolgál az Eltérő közlekedési rendek menüpont, ahol a megadott időintervallumra vonatkozóan felülírhatjuk az alapértelmezett rendet.

Az adatok átláthatóságának érdekében táblázatban jelennek meg az adatbázisból lekért rekordok.
Az adminisztrátor a keresés funkció segítségével kereshet az adatok attribútumai között, illetve sorba rendezheti azokat a gyorsabb kezelés érdekében.






















%5. fejezet
\chapter{Fejlesztői dokumentáció}
\label{fejl}
Az elkészült rendszer szerver-kliens architekturára épül.
Az architektúra modellje \aref{fig_5/architektura}.\ ábrán látható. 
A szerver felelős az adatok kezeléséért, mely a Symfony keretrendszerrel lett megvalósítva. 
Az adatokat a Doctrine keretrendszer segítségével éri el az adatbázisból. 
A rendszerben található a szerver, ami egy adminisztrációs feladatokat betöltő weboldal, és a kliens, egy Android alkalmazás, amely a felhasználók részére készült. 
Az adminisztrációs weboldalon az adatok a Twig sablonkezelő segítségével jelennek meg. 
Az Android-os kliens a kért adatokat JSON formrmátumban kapja meg. 
A továbbiakban részletesen bemutatásra kerül a szerver és a kliens felépítése, működése. 

% kell architektúra kép meg megtalálni az authenticationprocess képet

\section{A szerver}
\label{szerverfelepites}

A szerver a Symfony keretrendszer 2.8-as verziójára épült. 
A beérkező kérés áthalad a Symfony tűzfalán, ami elindítja az autentikációt. 
Ha sikeres volt az autentikáció, a kérés a megfelelő kontroller osztályhoz kerül feldolgozás után. 

\Picture{Az autentikáció folyamata}{5/authenticationprocess}{width=14cm}
% tűzfal, autentikáció, ilyesmi...

A kontroller osztályban a beírt URL-hez tartozó metódus hívódik meg. 
Alapesetben ezek az osztályok a \Highlight{AppBundle\textbackslash Controller} névtérben találhatóak. 
Az URL-címek metódusokra való leképzése kétféleképp valósítható meg a Symfony keretrendszer használatával. 

\begin{itemize}
	\item routing.yml fájlban való definiálás
	\item kontroller osztályokban annotációként
\end{itemize}

Az első opció akkor előnyös, ha útvonal alapján akarjuk meghatározni, hogy mely kódrészlet fog végrehajtódni a kontrollerek megfelelő kialakításával és az annotációk jó elhelyezésével. 
Másrészről a második megoldás itt előnyösebb lehet, mert ilyenkor a kódot látva a fájl elhagyása nélkül meg tudjuk állapítani, hogy milyen esetekben fog a kódrészlet meghívásra kerülni. 
A metódusok annotációi között megszabhatunk különböző paramétereket is, melyet az alábbi kódrészlet demonstrál:

\begin{lstlisting}
    /**
    * @Route("/{id}/edit", name="line_edit")
    * @Method({"GET", "POST"})
    */
    public function editAction(Request $request, Line $line)
    {
        // ...
    }
\end{lstlisting}

\subsection*{Autentikáció}
\label{authentication}

Az autentikáció a Symfony tűzfal moduljával történik. 
Az ehhez szükséges beállítások a \Highlight{app\textbackslash config} mappa \Highlight{security.yml} konfigurációs fájljában találhatóak. 
A szerverhez jelenleg két fajta felhasználó van rendelve, melyek az előbbiekben említett fájlban vannak definiálva.
A szerver feladatköréből kiindulva a jelenlegi igényeket ez a két felhasználó kielégíti, ezért nincs szükség újabb felhasználók hozzáadására. 
Ez azt jelenti, hogy a felhasználókhoz tartozó jelszavakat is ebben a fájlban tároljuk. 
Biztonsági szempontokból kritikus tényező, hogy ezek a jelszavak valamilyen módon titkosítva legyenek. 
A titkosításhoz a bcrypt algoritmust használtam fel. 
A Symfony segítségével parancssorból lehetséges bármilyen jelszót titkosítani a bcrypt algoritmussal. 

\begin{lstlisting}
php bin/console security:encode-password
\end{lstlisting}

A parancs futása közben bekéri, mi az a jelszó, amit titkosítani kell. 
Miután megadtuk a jelszót, a bemeneten lefuttatja az algoritmust és végül kiírja a titkosított karaktersorozatot. 
Ezt a szót megadva a fájlban jelszóként, továbbra is a titkosítatlan jelszóval be tud lépni a felhasználó az oldalra, viszont nem áll fent tovább a veszély, hogy illetéktelenek kezébe jutna a jelszó. 
\todo{hitvatkozás a hashre}

\subsection*{Autorizáció}
\label{authorization}

Az autorizáció folyamata szintén a Symfony tűzfal moduljának segítségével valósult meg. 
Ez a sikeres bejelentkezést követően hajtódik végre. 
A folyamat célja, hogy a bejelentkezett felhasználó csak a számára kijelölt szolgáltatásokat, oldalakat érje el. 
Az előző pontban említett két felhasználóhoz tehát az alábbi két szerepkör tartozik:

\begin{itemize}
	\item Látogató
	\item Adminisztrátor
\end{itemize}

A látogató szerepkörhöz tartozó felhasználónak csak az adatbázis-entitások lekérdezéséhez van jogosultsága. 
Az adminisztrátori engedéllyel rendelkező viszont a weboldal minden szolgáltatásához hozzáfér. 

A szerepkörökhöz hozzá lettek rendelve meghatározott formájú URL címek. 

\begin{lstlisting}
access_control:
    - { path: /json$, roles: ROLE_USER }
    - { path: ^/, roles: ROLE_ADMIN }
\end{lstlisting}
Az alábbi kódrészlet azt mutatja be, hogy a látogatói jogkör csak a \Highlight{\/json} végű URL-címekhez ad elérést, míg egy adminisztrátor minden címet elér. 
Ugyanezt a funkcionalitást meg lehet valósítani szintén annotációkkal is, de a weboldal kialakítása miatt célszerűbb volt ezt a megoldást választani, hisz így egyetlen fájlban elegendő meghatározni a hozzáférési szabályokat, annotációk használatával viszont minden kontroller osztályban egyenként kellett volna meghatározni ugyanazokat a szabályokat.
\subsection*{Adatbázis kapcsolat}
\label{adatbazis}
Az adatbázist a Doctrine keretrendszert használva éri el a szerver. 
Az alkalmazáshoz tartozó entitások helye a \Highlight{AppBundle\textbackslash Entity} névtér. 
Ezek mindegyike egyszerű PHP osztályok, a felhasznált keretrendszer alakítja ezeket az adatbázis tábláivá. 
Az osztályban szereplő adatmezőkhöz meg kell adni az annotációkat, hogy az átalakítás milyen módon történjen meg. 
Az entitások egymás közti kapcsolatait a következő annotációk jelzik:
\begin{itemize}
	\item OneToOne
	\item OneToMany
	\item ManyToOne
\end{itemize}
Az adatbázisban található adatok lekérdezés és objektumba való alakítása a \Highlight{Repository} osztályok segítségével történik. 
Ezek az osztályok a Doctrine \Highlight{EntityRepository} osztályából származnak, mely már rendelkezik a legalapvetőbb lekérdezésekkel, mint az azonosító vagy valamilyen tulajdonság alapján történő adatlekérés. 
Az összetettebb lekérdezésekhez a DQL nyelvet használtam fel, mely az entitások kapcsolatait az entitáshoz tartozó osztályban megadott annotációk alapján kezeli. 
Amennyiben az adatbázisból lekérdezett adatok módosultak, ahhoz, hogy ez a változás megmaradjon el kell menteni az adatbázisba. 
A Doctrine keretrendszerben található automatikus megoldás, mégpedig az EntityManager osztály, amely érzékeli a változást és elmenti a \Highlight{flush()} metódus használatával. 
Ha új adatot akarunk az adatbázishoz adni, azt is az EntitiyManager osztállyal tehetjük meg. 
Az elkészült új objektumot paraméterként átadva az osztály \Highlight{persist()} metódusának, a Doctrine keretrendszer úgy kezeli a továbbiakban az objektumot, mintha az az adatbázisból lekérdezett lenne, és így a \Highlight{flush()} metódussal az adatbázisba tudja írni. 
\section{Szerver feladatai}
\label{serverjob}
Az alábbiakban bemutatásra kerülnek a szerver főbb feladatai.
\subsection*{Járatok}
\label{lineinfodetails}
A városban közlekedő buszok más-más végpontok között közlekednek. 
Előfordulhat, hogy ugyanolyan számmal ellátott busz ugyanazon végpontok között közlekedik, de a köztes megállók eltérhetnek, vagy a köztes megállók ugyanazok, de a végpontok mások. 
Ezeknek az információk összefogására létrehozott \Highlight{LineInfo} entitás felelős. 
Ez az entitás tárolja a két végpontot, a közöttük közlekedő buszjáratot, és a járat típusát. 
A végpont és a buszjáratok szintén entitások, az ezek között fennálló kapcsolat az \ref{fig_5/lineinfo}. ábrán látható. 
\Picture{A járat entitás kapcsolatai}{5/lineinfo}{width=14cm}
\subsection*{Útvonalak}
\label{linehasstationsdetails}
A buszjáratok útvonaluk bejárása során több, köztes megállót is érintenek. 
Meg kell határoznunk, hogy egy adott busz milyen irányban haladva mely köztes megállókban áll meg. 
Ezt a feladatot a \Highlight{LineHasStations} entitás végzi. 
Az entitás segítségével meg tudjuk határozni, hogy az előző pontban vázolt járat egy példányának a végpontjain kívül milyen más állomásai vannak. 
Továbbá tároljuk a köztes megálló útvonalon belüli sorszámát, melyből megállapítható az útvonalon található állomások helyes sorrendje. 
Minden egyes objektum esetén tárolásra kerül még, hogy az előző megállóból milyen gyorsan tudunk eljutni. 
Természetesen a busz indulópontjának ezen attribútuma nulla. 
\Aref{fig_5/linehasstations}.\ ábrán megtekinthető milyen kapcsolatokat tartalmaz az útvonal entitás. 
\Picture{A járathoz kapcsolatai}{5/linehasstations}{width=14cm}
\subsection*{Indulási idők, különleges időpontok}
\label{starttimes}
Útvonaltervezés szempontjából fontos információ továbbá, hogy az adott járat az útvonalán található megállóiba milyen időpontban érkezik meg, továbbá különleges alkalmak esetén ezek hogyan változnak meg. 
Az indulási időpontot egy Starttime nevű entitás tartalmazza, melyben tárolódik továbbá az is, hogy ez az indulás mely járatra vonatkozik, valamint milyen típusú dátum esetén értetendő. 
A különböző típusú dátumokat külön entitás tartalmazza. 
Az alkalmazáshoz tovább tartozik egy Holidays entitás is, melyben az olyan, előre nem tervezett időpontokat tartalmazza amelyek befolyásolják bizonyos járatok indulási rendjét és esetleges útvonalát is. 
Az előbbiekben bemutatott entitások közötti kapcsolatot a \ref{fig_5/starttimes}. ábra mutatja be. 
\Picture{Különleges és indulási időpontok közötti kapcsolat}{5/starttimes}{width=14cm}

\section{Az Android kliens}
\label{androidclient}
Az alkalmazás fejlesztéséhez az Android Studio fejlesztői környezetet használtam.
Az Android Studio a Gradle rendszerre épül, amelynek számos funkciója megkönnyíti az Androidos alkalmazások fordítását.
A Gradle Androidos bővítményében található több Android specifikus parancs, amelyek közül az \Highlight{assemble} paranccsal készíthetjük el a futtatható és telepíthető állományt.
A futtatható fájl elkészítéséhez szükséges információkat a build.gradle fájlban adhatjük meg.
Ebben találhatóak a szükséges függőségek, amiket a Gradle bővítménye letölt, és belecsomagol az .apk-ba.
Továbbá itt található az applikáció által támogatott Android verziók száma.
Az általam készített alkalmazás futtatásához legalább a \Highlight{minSdkVersion 14} verzió szükséges, ami az Android operációs rendszer 4.0 verziójának felel meg.
Az \Highlight{AndroidManifest.xml} fájl az alkalmazás alapvető jellemzőit tartalmazza.
Ilyen jellemzők többek között a szükséges engedélyek és a Google Maps-hez tartozó metainformáció.
Az alkalmazás telepítéséhez és futtatásához szükség van engedélyekre a felhasználó részéről.
Ezek az engedélyek a következők:
\begin{lstlisting}
<uses-permission android:name="android.permission.INTERNET" /> 
<uses-permission android:name="android.permission.ACCESS_FINE_LOCATION" />
<uses-permission android:name="android.permission.ACCESS_NETWORK_STATE" />
\end{lstlisting}

Az első hozzáférés az Internettel való kommunikációhoz szükséges.
Szükségünk van továbbá olyan engedélyre amivel a Saját pozíció lekérhető és engedélyezhető a GPS a telefonon.
Mivel az alkalmazásnak szüksége van Internet hozzáférésre az adatok letöltéséhez, ezért az alkalmazás engedélyt kér a telefon hálózati állapotának lekéréséhez is.

\subsection*{Helyi adatbázis}
\label{localdatabase}
Fontos szempont volt helyi adatbázist létrehozni, mivel az alkalmazás és a szerver között folyamatosan felépülő kapcsolat sok erőforrást igényel.
Továbbá az olyan eseteknél, amikor a készülék nincs hálózatra kapcsolva, akkor nem érné el a szervert, így nem töltődnének le a működéséhez szükséges adatok.
Ezek következtében szükség volt arra, hogy az applikáció offline üzemmódban is képes legyen funkcionálisan működni.

Megvizsgáltam több Android platformhoz elérhető ORM (object-relational mapping vagyis objektum-relációs leképezés) keretrendszert, ezen belül is az ORMLite keretrendszert. 
Az ORMLite akkor előnyös, ha nem létezik adatbázis séma, amihez alkalmazkodnia kell.
Mivel az alkalmazás esetében már a szerver oldalon kialakításra került egy séma, emiatt a keretrendszer használata nehézkes volt.
Ebben a kontextusban szükséges volt, hogy manuálisan személyre szabhassam a helyi adatbázis sémáját, így végül a natív SQL utasítok használata mellett döntöttem.

Az androidos kliens implementálását követően kiadásra került több architekturális felépítést segítő könyvtár.
Ezek közül a \Highlight{Room} nevű könyvtár az adatok tárolásáért felel.
A \Highlight{Room} egy adatbázis absztrakciós könyvtár, amely elrejti az alatta található adattárolós részleteit.
Fordítási időben validálja az SQL lekérdezéseket, így ha ezek között található olyan, amely nem felel meg az adatbázis sémának, fordítási hibaként jeleníti meg futás idejű hiba helyett.

A helyi adatbázis-kezelő rendszer az SQLite, amely az Android operációs rendszerbe gyárilag beépített adatabázis-kezelő motor.
Mivel az SQLite relációs adatbázis, ezért a kommunikáció szimpla SQL utasításokkal történik.
Az applikáció sikeres működéséhez szükséges egy \Highlight{SQLiteOpenHelper} osztályból örököltetett osztály, amelyben az adatbázis nevét, verzióját és az adatbázis létrejöttekor végrehajtandó utasításokat tároljuk.
Az ősosztályból kettő metódust kötelező implementálni, az onCreate és az onUpgrade metódust.
Az onCreate metódus segítségével van lehetőség táblákat létrehozni az éppen elkészült adatbázisba, amit az alábbi kódrészlet szemléltet.
\begin{lstlisting}
@Override
public void onCreate(SQLiteDatabase sqLiteDatabase) {
	try {
		sqLiteDatabase.execSQL(DateTypeEntry.CREATETABLE);
		/* Create other tables similar to the above one. */
	} catch (SQLiteException e) { /*...*/ }
}
\end{lstlisting}
Ugyanakkor az onUpgrade metódus abban az esetben fut le, ha megnöveltük az adatbázis verzióját.
Ez a verziószám növekedés azt jelzi, hogy olyan változás történt az adatbázisban, amely visszafele nem kompatibilis, és egy transzformáció szükséges a régi és az új séma között.
Az \Highlight{SQLiteOpenHelper} osztály az SQL utasításokat tranzakcionálisan hatja végre.
Egy tranzakció tartalmazhat több SQL utasítást is, amelyek végrehajtása egyszerre történik.
Ha egy SQL utasítás sikertelen, akkor az összes utasítás visszavonásra kerül.

Az adatbázis minden táblájához egy statikus osztály lett létrehozva, melyek egy DatabaseContract nevű osztályban tárolódnak.
Minden belső osztály implementálja a BaseColumns osztályt, melynek köszönhetően megkapják az egyedi azonosításra használatos  \_id mezőt.
Erre egy példa az alábbi kódrészletben látható.
A további attribútumokat is implementálásra kerültek az adott osztályokban, majd a mezők alapján elkészült a tábla létrehozásához szükséges SQL parancs.
\begin{lstlisting}
static class DateTypeEntry implements BaseColumns{
	static final String TABLE_NAME = "date_type";
	static final String COLUMN_NAME_TYPE_NAME = "type_name";
	static final String COLUMN_NAME_ENABLED = "enabled";
	static final String CREATETABLE = "CREATE TABLE IF NOT EXISTS "+ TABLE_NAME+"( `"+_ID+"` INTEGER PRIMARY KEY NOT NULL, `"+COLUMN_NAME_TYPE_NAME+"` TEXT NOT NULL, `"+COLUMN_NAME_ENABLED+"` INTEGER DEFAULT 1);";
}
\end{lstlisting}

Az applikáció lokális adatbázisában lévő adatok POJO-ban tárolódnak, minden táblához külön POJO tartozik.
A POJO egy olyan Java objektum, amely nem rendelkezik speciális tulajdonsággal.

Ahhoz, hogy az alacsonyabb szintű SQL utasítások elkülönüljenek az üzleti logikától, létrehoztam egy \Highlight{DataManager} nevű absztrakt osztályt, melynek típus paramétere egy POJO objektum. 
Mivel az osztály absztrakt - így nem példányosítható - ezért ebben az osztályban kaptak helyet az olyan metódusok, melyek minden típusú POJO-ra általános érvényességűek. 
Az alábbi kódrészletben a DataManager osztály implementációja látható, ahol a createOrUpdate és a queryForAll metódusok az adatbázis tábláinak kezeléséért felelősek.
\begin{lstlisting}
public abstract class DataManager<T> {	
    DatabaseHelper helper;
	
	/* ... */
    public abstract void createOrUpdate(T data);
    public abstract List<T> queryForAll();
}
\end{lstlisting} 
Ebből az osztályból származnak a specifikus osztályok, melyeken keresztül adott típusú POJO-k kerülnek lementésre vagy visszakérdezésre a helyi adatbázisból.
Az ősosztályban található metódusokat mindenképp implementálnia kell minden származtatott osztálynak. 
Az olyan funkcionalitások esetén - amelyek nem minden POJO esetén voltak értelmezhetőek - a specifikus osztályban kerültek definiálásra.
Az alkalmazás mindenhol ezeket a manager osztályokat használja, ahol az adatbázissal való interakció szükséges.

\subsection*{Szerver-kliens kapcsolat}
\label{serverclient}
A helyi adatbázisban található adatok csak egy adott idő-intervallumban érvényesek, a menetrend változtatásával érvényüket vesztik. 
Ezért bizonyos időközönként frissíteni kell a benne szereplő információkat. 
Az adatok aktualizálása az alkalmazás indításakor történik, erre szolgál a korábban bemutatott szerver, ahonnan az alkalmazás az aktuális információkat kapja.
A szerver REST üzenetekkel válaszol a beérkező kérésekre.
A kommunikációt a Retrofit nevű REST klienssel oldottam meg, melynek segítségével könnyű a szervernek kéréseket küldeni.
A \Highlight{NetworkManager} osztály képes lekérdezni, hogy az adott eszköz rendelkezik-e jelenleg bármilyen internet eléréssel. 
Ha a készülék nem rendelkezik internet eléréssel, akkor csak a helyi adatbázist használja.
Ellenkező esetben pedig elkezdődik az adatok frissítése az applikáció elindításával.
A minél kisebb adathasználat érdekében a kliens tárolja, hogy mikor történt az utolsó frissítés. 
Az applikáció indításkor elküldi a legutolsó frissítési dátumot a szervernek, a szerver pedig csak azokat az adatokat küldi vissza, amelyek újabbak, mint a kapott időpont.
Miután az adatok aktualizálása sikeresen megtörtént, a frissítés időponja felülírásra kerül a helyi adatbázisban.
Az IDownloader interfészt a A \Highlight{BaseDownloader} absztrakt osztály implementálja, melynek konstruktorában kerülnek inicializálásra a Retrofit számára szükséges osztályok.
Lehetőségünk van minden kérést személyre szabni. 
Ezt felhasználva juttatjuk el a szerver számára a felhasználó azonosítására szolgáló adatokat és a legutolsó frissítési időpontot.
Az IDownloader interfész egyetlen metódus-definíciót tartalmaz:
\begin{lstlisting}
public interface IDownloader {
    void download();
}
\end{lstlisting}
A \Highlight{BaseDownloader} osztály tartalmaz még egy saveToDatabase metódust, amely az adatok helyi adatbázisba történő lementésére alkalmas.
A \Highlight{BaseDownloader} osztályból származtatott gyerekosztályok implementálják az előzőekben említett két metódust.

A Retrofit konfigurálása során egy interfész került megvalósításra, amelyben metódus-definíció formájában felsorolásra kerülnek azok a hívások, amelyeket a Retrofit képes végrehajtani.
\begin{lstlisting}
public interface MyApiEndpointInterface {
	/* ... */
	@GET("stations/json")
    Call<List<Stations>> getStations();
}
\end{lstlisting}
A fenti példán látható, hogy a megállók eléréséhez annotáció formájában megadásra került az elérési útvonal.
Ennek visszatérési értéke egy olyan lista lesz, amely megállókat tartalmaz.
A listában található megállókat a Retrofit automatikusan szerializálta ki a szervertől kapott JSON válaszból.
A szerializáláshoz az alábbi entitásban található annotációs konfigurációt veszi alapul.
\begin{lstlisting}
public class Stations {
	/* ... */
	@SerializedName("lat")
	@Expose
	private double lat;
}
\end{lstlisting}

\subsection*{A felhasználó felület kialakítása}
\label{layout}
A kód karbantarthatóságának érdekében a felhasználói felület kialakítása az MVP (Modell-View-Presenter) szoftvertervezési mintára épül.
Implementálása a \Highlight{Mosby} könyvtár segítségével történt.
Előnye, hogy az üzleti logika elhatárolódik a felhasználó felülettől, melynek köszönhetően a kezelőfelület módosítható a logika megváltoztatása nélkül.
Az alkalmazás minden képernyő-nézetéhez tartozik egy úgynevezett activity.
Az activity a felhasználók és az applikáció közötti legfőbb alapelem.
A Mosby könyvtárban találhatóak előre definiált osztályok és felületek, melyek kiindulási pontként szolgálnak az MVP megvalósításához.
A tervezési minta alkalmazásához létre kell hozni minden activity esetén egy interfészt, melyben definiálásra kerül, hogy milyen műveleteket tud végrehajtani az activity.
Az üzleti logika tényleges megvalósítása a presenter osztályban történik, így ez az aspektus nem lesz kihatással a felhasználói felületre.
A presenter osztályok típusparaméterként megkapják az előzőekben említett felületet.
A létrehozott felületet implementálnia kell az activity-nek, és típusparaméterként meg kell adni mind a felületet, mind a presenter osztályt.
A felület, presenter és activity osztályok mind a könyvtárban található alaposztályok leszármazottjai.
Minden activity-hez tartozik egy .xml kiterjesztésű fájl, mely a felhasználói felület építő elemeit tartalmazza.
A tervezésre lehetőség van az XML fájlban történő szöveges kifejtésre, illetve grafikus felületen keresztül, amelyre drag-and-drop módszerrel helyezhetjük rá az elemeket.
Az előbb bemutatott tervező felület \aref{fig_5/designer}.\ ábrán látható.
\Picture{Google Maps útvonaltervezés}{5/designer}{width=10cm}

\subsection*{Google Maps integráció}
\label{googlemaps}
A Google Maps alkalmazásba való integrációját több előkészületi lépés előzte meg.
Első lépésként a Google Developer weboldalon egy projekt létrehozása szükséges.
A projektben engedélyezésre kerültek azon szolgálatások, amelyek az alkalmazás funkcionalitásához elengedhetetlenek.
Az engedélyezett szolgáltatások használatához az alkalmazásban egy hitelesító kulcsot kötelező megadni, amely a fent említett Google Developer oldalon generálódik. 
Mindezek után minden lehetőség adott a térkép megjelenítésére az alkalmazásban.
Ennek legegyszerűbb módja, ha a \Highlight{SupportMapFragment} osztályt használjuk.
%https://developers.google.com/android/reference/com/google/android/gms/maps/SupportMapFragment
Ez az osztály körülfog egy térképet és automatikusan kezeli annak életciklusát.
Egyszerűen megjeleníthető a kívánt activity-ben, az XML fájlhoz az alábbiakban bemutatott módon kell hozzáadni:
\begin{lstlisting}
<fragment
	android:id="@+id/map"
	android:name="com.google.android.gms.maps.SupportMapFragment"
	android:layout_width="match_parent>
	android:layout_height="match_parent"
/>
\end{lstlisting}

Ahhoz, hogy a térkép dinamikusan módosítható legyen, egy \Highlight{GoogleMap} példányra van szükség.
A \Highlight{SupportMapFragment} osztályon a \Highlight{getMapAsync(OnMapReadyCallback)} metódust meghívva az osztály inicializálja a térképet.
Miután ez megtörtént, a metódus paramétereként átadott \Highlight{OnMapReadyCallback} felületen meghívódik az \Highlight{onMapReady} metódust, ami egy \Highlight{GoogleMap} példányt biztosít.
A \Highlight{OnMapReadyCallback} felület lehet külön osztály, de akár egy activity is implementálhatja azt, így amikor a példány elérhető, azonnal interakcióba léphetünk vele az activity-ben.
A \Highlight{GoogleMap} példány segítségével a térképhez adhatjuk a saját pozíciót lekérő gombot vagy megadhatjuk a térkép nézetének típusát.

\subsection*{Útvonaltervező}
\label{planner}
A felhasználó az Útvonaltervezés menüpontot kiválasztva Veszprém városán belül két tetszőleges hely között tervezhet közlekedési járatokat is alapul véve.
Ehhez meg kell adnia a kiindulási és érkezési pontokat.
Ezt megteheti úgy, hogy a saját pozícióját megadva automatikusan kitöltődik az indulási pont és már csak az érkezési helyet kell megadni, vagy mindkettőt manuálisan kell felvinnie.
A felvitelt \Highlight{SupportPlaceAutocompleteFragment}-en keresztül teheti meg a felhasználó.
Ahogy a felhasználó elkezdi beírni a helyszínt az \Highlight{SupportPlaceAutocompleteFragment} automatikus kiegészíti és javaslatokat tesz.
Annak érdekében, hogy a kevésbé releváns helyeket ne ajánlja fel, meg lehet adni egy területet, melyben található helyneveket előnyben részesítse a kiegészítés alkalmával.
Az előnyben részesítendő területet a \Highlight{setBoundsBias} metódus segítségével jelölhetjük ki.

Ahhoz, hogy a két kijelölt hely között megkapjuk a lehetséges útvonalakat, le kell futtatnunk valamilyen útvonaltervező algoritmust.
A későbbi továbbfejlesztést elősegítendő létrehoztam egy \Highlight{IPlanner} nevű felületet, melynek egyetlen metódusa a \Highlight{calculate}.
A felület mellé elkészült még egy BasePlanner nevű absztrakt osztály is, amely megvalósítja a felületet.
A tényleges útvonaltervezést implementáló osztályoknak ebből kell leszármazniuk.
Az ősosztály rendelkezik minden olyan fontos információval, amire az útvonaltervezés közben szüksége lehet az algoritmusnak, például az indulási és érkezési pontok. 
Rendelkezik továbbá egy \Highlight{OnPlannerListener} nevű callback felülettel, melynek \Highlight{onPlannerFinished} metódusát akkor kell meghívni, amikor az algoritmus végzett a tervezéssel.

Elkészült egy \Highlight{AdvancePlanner} nevű útvonaltervező osztály, mely képes a két pont között útvonalat tervezni.
Tervezés közben számításba veszi a lehetséges útvonalak között, hogy mennyit kell a felhasználónak sétálni, mennyit kell utazni időben és távolságban, valamint hogy hányszor kell átszállnia.
Az algoritmus menete a következő:
\begin{enumerate}
	\item A két megadott helyhez megkeressük a hozzájuk közel található buszmegállókat.
	\item Ezután leellenőrizzük, hogy a két listában található-e olyan eset, amikor közvetlenül eljuthatunk a két ponthoz közeli megállókba egy járattal. 
	Amennyiben igen, ezeket a lehetőségeket hozzáadjuk a lehetséges útvonalakhoz.
	\item Ha nem található közvetlen járat, az átszállásos járatok vizsgálata történik meg.
\end{enumerate}

Az algoritmus eredményül egy olyan listát ad át a \Highlight{onPlannerFinished} metódusnak, melynek minden eleme több utasítást tartalmaz.
Négy fajta utasítást különböztet meg az alkalmazás: sétálás indulási ponttól megállóig, sétálás megállótól érkezési pontig, buszutazás és amennyiben szükséges átszállás esetén séta két megálló között.
Az algoritmus befejezése után egy új képernyő jelenik meg.
Itt megtekinthetőek az eredményül kapott javaslatok csoportokra bontva.
Jelenleg három csoport szerint vannak rendezve az útvonalak:
\begin{itemize}
	\item Legkevesebb utazási idő
	\item Legkevesebb gyaloglás
	\item Legkevesebb átszállás
\end{itemize}

Amint a felhasználó kiválasztja a számára optimális megoldást, visszatér az előző képernyőre, ahol már a kiválasztott útvonal vizuálisan is megjelenik, a térképen jelölve a gyaloglási útvonalat és a járat(ok) által érintett megállókat.
\newpage

%6. fejezet
\chapter{Továbbfejlesztési lehetőségek}
\label{tovabbfejl}
%6. fejezet

Az elkészült alkalmazás a felé támasztott követelményeknek eleget tesz, ennek ellenére funkcionalitása tovább bővíthető.
Fontos szempont, hogy a fejlesztés végeztével tesztelési céllal is ellenőrzésre kerüljön az applikáció.
Teszteléskor olyan hibák is jelentkezhetnek, amelyek a felhasználók által használt akciókból erednek.
Ennél a szakasznál lényeges lehet egy külső személy bevonása az ellenőrzésbe, ezáltal lecsökkentve az alkalmazás nem megfelelő használatából eredő hibák számát.
Mindemellett fontos, hogy az applikáció minél magasabb szinten elégítse ki a felhasználói igényeket, így továbbfejlesztésre és új funkciók bevezetésére is szükség lehet.

\section*{Android Room framework}
\label{androidroom}
Az alkalmazás egyszerűsége és fenntarthatóságának érdekében érdemes lenne az előző fejezetben ismertetett Room adatbázis absztrakciós könyvtárat használni.
Segítségével egyszerűbbé és átláthatóbbá válna a kódbázis.

\section*{További városok hozzáadása}
\label{morecity}
Veszprém megyében több kisváros is rendelkezik helyi tömegközlekedéssel, ami egy-kettő járattól akár húsz járatot is jelenthet.
Ebből kifolyólag a továbbiakban fontos lehet a környező városok tömegközlekedésének integrálása.
Ehhez az adatbázis átalakítása, továbbfejlesztése szükséges, hogy az általános érvényű legyen minden városra.
Ez a funkció segítséget nyújtana a kisebb városokban élőknek, illetve azoknak a turistáknak, akik a megye több városát felkeresve nyaralnak.

\section*{Többnyelvűség}
\label{internationalization}
Külföldről érkező turisták esetében megfontolandó a többnyelvűség támogatása.
Ebben az esetben lényeges lehet egy nyelvválasztó funkció implementálása az alkalmazásba.
Első sorban az angol és német nyelvek megvalósítása, de a későbbiekben akár még több nyelv bevonásával.

\section*{Útvonaltervezés továbbfejlesztése}
\label{routeplan}
Az alkalmazás optimálisabb működése érdekében fontos tényező lehet az útvonaltervezés továbbfejlesztése.
A felhasználói élmény növekedése érdekében a tervező algoritmus pontosabbá, gyorsabbá tétele.
Ehhez a jelenlegi megoldást le lehetne cserélni okosabb, heurisztikus algoritmusra.
Az útvonaltervező mellett a felhasználói felületet is tovább lehet fejleszteni, ilyen feladat lehet többek közt a járatok megjelenítésének pontosítása. 

\section*{Mobil értesítések}
\label{pushnot}
További ötletként felmerült, hogy a felhasználók számára értesítéseket küldjön ki az alkalmazás.
A felhasználó beállíthatná a tartózkodási helyét, illetve egy járat indulási időpontját, az alkalmazás pedig jelezné egy értesítés formájában, ha a felhasználónak indulnia kell a megadott helyről, hogy elérje a korábban beállított járatot.
Olyan esetben is hasznos lenne ez a funkció, ha útvonalváltozás történik egy buszjáratnál, így a felhasználó időben értesítve lenne a módosításról.

\section*{Widget}
\label{widget}
Érdemes lenne megvalósítani egy olyan modult, amely kitehető a telefon főképernyőjére.
Az alkalmazásban, a Kedvencek menüpont alatt beállított járatok táblázatai között válthatna a felhasználó, így az alkalmazás elindítása nélkül, hamar megtekinthető lenne az egyes járatok menetrendje.
Mindemellett fontos, hogy ez a modul is folyamatosan friss információkat jelenítsen meg, ha elérhető frissebb adat, az legyen jelezve a felhasználó felé is.
\newpage



%% <== End of hints
%%%%%%%%%%%%%%%%%%%%%%%%%%%%%%%%%%%%%%%%%%%%%%%%%%%%%%%%%%%%%


%%%%%%%%%%%%%%%%%%%%%%%%%%%%%%%%%%%%%%%%%%%%%%%%%%%%%%%%%%%%%
%% BIBLIOGRAPHY AND OTHER LISTS
%%%%%%%%%%%%%%%%%%%%%%%%%%%%%%%%%%%%%%%%%%%%%%%%%%%%%%%%%%%%%
%% A small distance to the other stuff in the table of contents (toc)
\addtocontents{toc}{\protect\vspace*{\baselineskip}}


\bibliographystyle{mybibstyle}
\bibliography{cite}

%% The List of Figures
%\clearpage
%\addcontentsline{toc}{chapter}{List of Figures}
%\listoffigures

%% The List of Tables
%\clearpage
%\addcontentsline{toc}{chapter}{List of Tables}
%\listoftables


\newpage

\Large
\begin{center}
	\textbf{MELLÉKLET}
\end{center}
\normalsize
\noindent
A mellékelt CD könyvtárszerkezete


% \begin{itemize}
%    \item \textbf{Dokumentum}
%    \begin{itemize}
%        \item \textbf{Forrás} - A szakdolgozat szerkeszthető formátumban
%        \item \textbf{Hivatkozások} - A szakdolgozatban lévő internetes hivatkozások letöltve
%        \item szakdolgozat.pdf
%    \end{itemize}
%    \item \textbf{Forrás} - A program forrásállománya
%    \begin{itemize}
%        \item \textbf{HexEngine}
%        \item \textbf{HexEngine-android}
%        \item \textbf{HexEngine-desktop}
%    \end{itemize}
%    \item \textbf{Program} - A futtatható program
%    \begin{itemize}
%        \item \textbf{bin} - A program grafikai és konfigurációs fájljai ami szükségesek az indításhoz
%        \item \textbf{hav} - A program a mentéseket tárolja itt
%        \item \textbf{java\_telepito} - A Java környezet telepítői
%        \item \textbf{libs} - Futtatáshoz kapcsolódó java fájlok
%        \item starter.jar
%    \end{itemize}
% \end{itemize}





%%%%%%%%%%%%%%%%%%%%%%%%%%%%%%%%%%%%%%%%%%%%%%%%%%%%%%%%%%%%%
%% APPENDICES
%%%%%%%%%%%%%%%%%%%%%%%%%%%%%%%%%%%%%%%%%%%%%%%%%%%%%%%%%%%%%
\appendix
%% ==> Write your text here or include other files.

%\input{FileName} %You need a file 'FileName.tex' for this.


\end{document}

