%3. fejezet

A fejezetben bemutatásra kerülnek a a funkciókövetelmények, amelyeket a fejlesztés előtt és közben figyelembe kellett venni, hogy az utazóközönség számára optimális alkalmazást készüljön el. 
Továbbá a rendszer másik szereplői, az adminisztrátorok számára is könnyen kezelhető, felhasználóbarát felület készüljön el. 
A .képen láthatóak a rendszer szereplői és a hozzájuk tartozó használati esetek. 


%3.1
\section{Követelmények}
\label{kovetelmeny}

Az alábbi követelmények tartalmazzák azokat a tulajdonságokat, amik az elkészült alkalmazás jellemzői között kell

\begin{itemize}
	\item Sebesség
	\\
	Válaszidő lecsökkentése minél alacsonyabb szintre. 
	A felhasználóknak fontos, hogy az alkalmazás segítségével gyorsan és megbízhatóan tudjanak információhoz jutni. 
	\item Alacsony erőforrás felhasználás
	\\
	Az egyik legfontosabb követelmény, hogy az alkalmazás kevés erőforrás igénybevételével is megfelelően működjön. 
	Ehhez szükség volt arra, hogy a telefonon egy lokális adatbázis üzemeljen. 
	Így az alkalmazás használatakor nem kell internetkapcsolatot biztosítani az adatok elérhetőek lesznek a telefon adatbázisából.	
	\item Megbízhatóság
	\\
	Létre kellett hozni egy olyan webes felületet, ahol az adatbázis kezelhető, változások esetén pedig módosíthatóak az adatok. 
	Ebből kifolyólag az alkalmazás mindig naprakészen szolgálja az információt a felhasználóknak.
	\item Könnyű kezelhetőség
	\\
	A felhasználók számára fontos, ha nincsenek bonyolult akciók, hanem lehetőleg az összes funkció használata egyértelmű, ezzel is gyorsítva az alkalmazás használatát. 
	Ha bizonyos jelölések, rövidítések igénylik, akkor súgót kell hozzáadni az alkalmazáshoz.
	
\end{itemize}


%3.2
\section{Funkcionális követelmények}
\label{funkckov}

A .ábrán láthatóak a felhasználóval és az adminisztrátorral kapcsolatos használati esetek, amelyeket a következőekben fogok részletesen kifejteni.

\section*{Android alkalmazás}
\begin{itemize}
	\item \textbf{Útvonaltervezés}
	\\
	Az útvonaltervezés az alkalmazás fő funkciója, ezért nagy hangsúlyt kellett a megvalósítására fektetni. 
	Könnyű kezelőfelületet kívánt, egyértelmű jelöléseket amik bárki számára egyszerűvé teszik a használatát. 
	Ez a menüpont azokat az utasokat célozza meg elsődlegesen, akik számára Veszprém ismeretlen terület. 
	Fontos volt, hogy különböző utazási módozatok is elérhetőek legyenek, például azoknak akik buszjegyet váltanak, azok minél kevesebb átszállással kínálja az alkalmazás az útvonalat. 
	Továbbá, mivel ez egy térképet integráló funkció, ezért megfelelő módon kellett megjeleníteni a térképet. 	
	\item \textbf{Menetrend}
	\\
	A menetrend funkció akkor kap szerepet egy felhasználónál elsősorban, amikor nincs elérhető internetkapcsolat a mobilkészülékén. 
	Ezen kívül azoknál akik rendelkeznek a tömegközlekedésről legalább alapszintű ismerettel, így jártasak a menetrendben és az indulási időkről szeretnének informálódni. 
	Megjelenésénél figyelni kellett arra, hogy felülete letisztult legyen, mégis a keresett információ könnyen megtalálható legyen. 	
	\item \textbf{Megállók}
	\\
	Az alkalmazásban szükség volt egy olyan szolgálatásra is, ahol a felhasználók tájékozódni tudnak a városban a megállókról. 
	Mivel előfordulhat olyan eset, hogy valaki tisztában van a megálló elhelyezkedésével, viszont meg szeretné tudni, hogy milyen járatok érintik anélkül hogy az egész menetrendet át kéne olvasnia, ezért fontos volt implementálni egy ilyen funkciót az alkalmazásba. 
\end{itemize}
\section*{Weboldal}
\begin{itemize}
	\item Buszjáratok kezelése
	\item Menetidők kezelése
	\item Megállók kezelése
	\item Ideiglenes járatmodosulások kezelése
\end{itemize}
	
	

%3.3
\section{Felhasznált technológiák}
\label{felhaszntech}
\section*{Térkép}
Egy útvonaltervező alkalmazás nagyon fontos tulajdonsága a megjelenés. 
Egy ilyen alkalmazásnál alapvető elvárás, hogy a térképen megjelenő információ könnyen kiolvasható legyen, hogy a felhasználók könnyedén el tudják választani a lényeges információt azoktól, amelyek nem szükségesek a tájékozódástól. 
Emiatt szükségem volt egy olyan térképes szolgáltatásra, ami az előbbi céloknak megfelel. 
Mielőtt megkezdtem a fejlesztést, több ilyen szolgáltatást is megvizsgáltam, abból a célból, hogy a legmegfelelőbbet tudjam kiválasztani a tapasztalatok alapján a szakdolgozatomhoz. 
Az egyik legfontosabb szempont a kiválasztásnál az volt, hogy ingyenesen elérhető legyen a szolgáltatás. 
A kutatás során két fő jelöltre sikerült leszűkítenem a listát, a Google Maps-re illetve az OpenStreetMap-re. 
Mivel az általam készített programot Android platformra terveztem elkészíteni, ezért olyan térképes API-ra volt szükségem, amit be lehet építeni Android applikációba, és ennek a célnak mind a kettő szolgáltatás megfelelt. 
Mindezek mellett fontos volt számomra, hogy olyan térkép alkalmazást válasszak, ami felhasználói körökben jól ismert. 
Emiatt megvizsgáltam, hogy az alkalmazásoknak mekkora a felhasználói köre. 
Az OpenStreetMap Android alkalmazása körülbelül 5 millió  felhasználóval rendelkezik, míg a Google Maps letöltése meghaladja az 1 milliárd felhasználót. 
Ezekből az adatokból következtetni tudtam arra, hogy a felhasználók szélesebb köre miatt a Google Maps felülete sokkal szélesebb körben ismert az emberek között.

\Picture{A Google Maps és az OSM összehasonlítása}{3/terkeposszehas}{width=10cm}

A \ref{fig_3/terkeposszehas}.ábrán látható szempontok figyelembe vételével végül a Google Maps-re esett a választásom, mint integrálható térképes szolgáltatás.

\section*{Android, Java}

\section*{Symfony}
\section*{MySQL}















