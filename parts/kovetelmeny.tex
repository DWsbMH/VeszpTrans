%3. fejezet

A fejezetben bemutatásra kerülnek a funkciókövetelmények, amelyeket a fejlesztés előtt és közben figyelembe kellett venni, hogy az utazóközönség számára egy mai igényeket kielégítő, modern alkalmazás készüljön el.  
Cél volt továbbá, hogy a rendszer másik szereplői, az adminisztrátorok számára is könnyen kezelhető, felhasználóbarát felület valósuljon meg.

%3.1
\section{Követelmények}
\label{kovetelmeny}

Az alábbi követelmények tartalmazzák azokat a tulajdonságokat, amiket az elkészült alkalmazásnak tartalmaznia kell.

\begin{itemize}
	\item Sebesség
	\\
	Válaszidő lecsökkentése minél alacsonyabb szintre. 
	A felhasználóknak fontos, hogy az alkalmazás segítségével gyorsan és megbízhatóan tudjanak információhoz jutni. 
	\item Alacsony erőforrás felhasználás
	\\
	Az egyik legfontosabb követelmény, hogy az alkalmazás kevés erőforrás igénybevételével is megfelelően működjön. 
	Ehhez szükség volt arra, hogy a telefon egy lokális adatbázist üzemeltessen. 
	Így az alkalmazás használatakor nem kell internetkapcsolatot biztosítani az adatok elérhetőek lesznek a telefon adatbázisából.	
	\item Megbízhatóság
	\\
	Létre kellett hozni egy olyan webes felületet, ahol az adatbázis kezelhető, változások esetén pedig módosíthatóak az adatok. 
	Ebből kifolyólag az alkalmazás mindig naprakészen szolgálja az információt a felhasználóknak.
	\item Könnyű kezelhetőség
	\\
	A felhasználók számára fontos, ha nincsenek bonyolult akciók, hanem lehetőleg az összes funkció használata egyértelmű, ezzel is gyorsítva az alkalmazás használatát. 
	Ha bizonyos jelölések, rövidítések igénylik, akkor súgót kell hozzáadni az alkalmazáshoz.
	
\end{itemize}


%3.2
\section{Funkcionális követelmények}
\label{funkckov}

Ebben a alfejezetben a felhasználóval és az adminisztrátorral kapcsolatos használati esetek kerülnek kifejtésre.

\section*{Android alkalmazás}
\begin{itemize}
	\item \textbf{Útvonaltervezés}
	\\
	Az útvonaltervezés az alkalmazás fő funkciója, ezért nagy hangsúlyt kellett a megvalósítására fektetni. 
	Könnyű kezelőfelületet kívánt, egyértelmű jelöléseket, amik bárki számára egyszerűvé teszik a használatát. 
	Ez a menüpont azokat az utasokat célozza meg elsődlegesen, akik számára Veszprém ismeretlen terület. 
	Fontos volt, hogy különböző utazási módozatok is elérhetőek legyenek, például azoknak, akik buszjegyet váltanak, minél kevesebb átszállással kínálja az alkalmazás az útvonalat. 
	Továbbá, mivel ez egy térképet integráló funkció, ezért megfelelő módon kellett megjeleníteni a térképet. 	
	\item \textbf{Menetrend}
	\\
	A menetrend funkció akkor kap szerepet egy felhasználónál elsősorban, amikor nincs elérhető internetkapcsolat a mobilkészülékén. 
	Ezen kívül azoknál, akik rendelkeznek a tömegközlekedésről legalább alapszintű ismerettel, így jártasak a menetrendben és az indulási időkről szeretnének informálódni. 
	Megjelenésénél figyelni kellett arra, hogy felülete letisztult, könnyen kezelhető és informatív legyen.
	\item \textbf{Megállók}
	\\
	Az alkalmazásban szükség volt egy olyan szolgálatásra is, ahol a felhasználók tájékozódni tudnak a városban a megállókról. 
	Mivel előfordulhat olyan eset, hogy valaki tisztában van a megálló elhelyezkedésével, viszont meg szeretné tudni, hogy milyen járatok érintik anélkül hogy az egész menetrendet át kéne olvasnia, ezért fontos volt implementálni egy ilyen funkciót az alkalmazásba. 
\end{itemize}
\section*{Weboldal}
\begin{itemize}
	\item Buszjáratok kezelése
	\item Menetidők kezelése
	\item Megállók kezelése
	\item Ideiglenes járatmodosulások kezelése
\end{itemize}
	
	

%3.3
\section{Felhasznált technológiák}
\label{felhaszntech}
\section*{Térkép}
Egy útvonaltervező alkalmazás nagyon fontos tulajdonsága a megjelenés. 
Egy ilyen alkalmazásnál alapvető elvárás, hogy a térképen megjelenő információ könnyen kiolvasható legyen, hogy a felhasználók könnyedén el tudják választani a lényeges információt azoktól, amelyek nem szükségesek a tájékozódáshoz. 
Emiatt szükségem volt egy olyan térképes szolgáltatásra, ami az előbbi céloknak megfelel. 
Mielőtt megkezdtem a fejlesztést, több ilyen szolgáltatást is megvizsgáltam, abból a célból, hogy a legmegfelelőbbet tudjam kiválasztani a tapasztalatok alapján a szakdolgozatomhoz. 
Az egyik legfontosabb szempont a kiválasztásnál az volt, hogy ingyenesen elérhető legyen a szolgáltatás. 
A kutatás során két fő jelöltre sikerült leszűkítenem a listát, a Google Maps-re illetve az OpenStreetMap-re. 
Mivel az általam készített programot Android platformra terveztem elkészíteni, ezért olyan térképes API-ra volt szükségem, amit be lehet építeni Android applikációba, és ennek a célnak mind a kettő szolgáltatás megfelelt. 
Mindezek mellett fontos volt számomra, hogy olyan térkép alkalmazást válasszak, ami felhasználói körökben jól ismert. 
Emiatt megvizsgáltam, hogy az alkalmazásoknak mekkora a felhasználói köre. 
Az OpenStreetMap Android alkalmazása körülbelül 5 millió  felhasználóval rendelkezik, míg a Google Maps letöltése meghaladja az 1 milliárd felhasználót. 
Ezekből az adatokból következtetni tudtam arra, hogy a felhasználók szélesebb köre miatt a Google Maps felülete sokkal szélesebb körben ismert az emberek között.

\Picture{A Google Maps és az OSM összehasonlítása}{3/terkeposszehas}{width=12cm}

\Aref{fig_3/terkeposszehas}.\ ábrán látható szempontok figyelembe vételével végül a Google Maps-re esett a választásom, mint integrálható térképes szolgáltatás.

\section*{Android, Java}
Az Android platformon\cite{androidprogramming} futó alkalmazások elsődleges programozási nyelve a Java.
A Java platformfüggetlen és széles körben elterjedt, a Sun Microsystems által fejlesztett nyelvet az 1990-es évek óta folyamatosan fejlesztik.
Az Android egy Linux alapú operációs rendszer, amely több pozitívummal is rendelkezik, ilyen például a hordozhatóság és a biztonság.
Futtatási sebessége a Java-hoz képes javult, hiszen a kibővített Java programozási nyelvet Android esetén egy olyan virtuális gépen futtatják, amely szerves részét képezi az Android operációs rendszernek. 
\todo{telenyomni hivatkozásokkal a részt}

\section*{Symfony}
A Symfony egy PHP alapú keretrendszer.
A vele készült webes alkalmazások könnyen karbantarthatóak és jól skálázhatóak, mivel a keretrendszer az MVC tervezési mintára épül.
A Symfony felépítése moduláris, ennek köszönhetően könnyen bővíthető.
Továbbá a Symfony egy-egy komponensét más keretrendszerrel készült alkalmazásban külön is használhatjuk, amit szintén a moduláris szerkezet tesz lehetővé.
Minden beérkező kérést a keretrendszer dolgoz fel, melyek az útvonalválasztás után meghívják a megfelelő kontrollerek metódusát.
Továbbá a kontrollerek hozzáférnek a services.yml fájlban definiált szolgáltatásokhoz is.
Az itt definiált szolgáltatások példányait, a Symfony dependency injection segítségével juttatja el a kontrollerekhez.
A Symfony segítségével a dinamikus tartalom is könnyen átalakítható a benne található template nyelvek egyikének köszönhetően.
Az általam választott nyelv a twig, amellyel a megjelenítés elválasztható az üzleti logikától. 

\section*{MySQL}
A MySQL  egy SQL (Structured Query Language, vagyis struktúrált lekérdező nyelv) alapú relációs adatbázis-kezelő szerver.
Teljesen nyílt forráskodú, feltételezhetően emiatt terjedt el széles körben.
Egyszerűen használható és költséghatékony megoldást nyújt dinamikus webhelyek számára.
Az adatbázis kezelésére a phpMyAdmin adminisztrációs eszközt használtam, amely segítségével Interneten keresztül menedzselhettem az adatokat.

\section*{Doctrine}
A Doctrine keretrendszer az adatbázisban található adatok elérésére alkalmas.
A segítségével PHP osztályok képezhetőek le relációs adatbázis táblákká.
A keretrendszer lehetőséget ad arra, hogy az alkalmazás adatbázisa különösebb ráfordítás nélkül lecserélhető legyen.
A DQL, vagyis a Doctrine lekérdező nyelv használatával lehet az adatokat elérni.
A Doctrine véd továbbá az SQL injektálásos támadásokkal szemben is.
A Doctrine úgynevezett Repository-kat használ az adatok lekérdezésére, amelyek lehetővé teszik, hogy alkalmazás logikája és a lekérdezés megvalósítása elvonatkoztatható legyen egymástól.
Továbbá azért esett a választásom a Doctrine-ra, mert a keretrendszer integrálva van a Symfony-ba, illetve szolgáltatásként elérhető a Doctrine EntityManager objektuma.
Új entitások generálásában és az adatbázis tábláinak frissítésében is segítséget nyújt a Symfony parancssoros eszköze.













