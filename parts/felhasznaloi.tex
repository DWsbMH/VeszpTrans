%4. fejezet

A 4.fejezetben bemutatásra kerül a szakdolgozat keretében fejlesztett Android alkalmazás felhasználói szemszögből.
Ebben kifejtésre kerül milyen menüpontok találhatóak az applikációban, és ezek milyen funkcionalitással rendelkeznek.
Továbbá ismertetem az adatok menedzselésére szolgáló weboldal felületét és működését, amely az admin felhasználók munkáját teszi könnyebbé.

%4.1
\section{Android alkalmazás}
\label{androidapp}}
%képek: applogo, főmenü
Az applikáció ikonja egy Veszprémben járműparkjában is szereplő autóbusz, az Ikarus 280.
Az alkalmazás indítása után az alkalmazás lekéri az adatbázisból azokat az adatokat, amik módosítási dátuma későbbi, mint az utolsó letöltés ideje.
Ezt a felhasználó egy felugró ablak képében látja, amelyen az alkalmazás közli, hogy frissítés van folyamatban, és kéri a felhasználók türelmét.
Amint az adatok aktualizálása befejeződött, a felugró ablak eltűnik, átadva helyét a főmenü menüpontjainak.

\section {Főmenü}
\label {fomenu}
A főmenü az alkalmazás nevét és négy almenüt foglal magába:
\begin{itemize}
	\item Útvonaltervezés
	\\
	\item Menetrendek
	\\
	\item Megállók
	\\
	\item Kedvencek
	\\
\end{itemize}

A . ábrán látható az alkalmazás főmenüjének kiosztása. 
Az applikáció témája a lila szín és annak árnyalatai, amely lehetővé teszi a felhasználók figyelmének felkeltését, mégis letisztult külsőt kölcsönöz.

\section {Útvonaltervezés}}
\label {utvonalterv}
Az Útvonaltervezés menüpontot kiválasztva egy Google Maps térkép jelenik meg Veszprém városára fókuszálva.
A képernyőn ezen kívül egy beviteli mező és három gomb található.
A beviteli mezőt a felhasználók a térképen való keresésre használhatják.
A keresés fő fókusza Veszprémre irányul, a mező automatikusan próbálja kiegészíteni a begépelt helyszínt, Veszprém környékére összpontosítva, ahogy az a . ábrán is látható.
Az alkalmazás képes más földrajzi helyre irányuló kereséseket is kiegészíteni, viszont ez a funkció - az alkalmazás minél effektívebb működése érdekében - le van korlátozva.
%kép: Autocomplete, találat markeres
Ahogy a . ábrán is látszik, a találatot az alkalmazás egy úgynevezett 'marker' lerakásával jelöli a térképen.
A bal oldalon található gombbal a felhasználó képes a térkép nézetének megváltoztatására.
Az alapérmezett mód a domborzati megjelenés, a gomb megnyomávásával átválthatunk műholdas nézetre.
A jobb oldalon két gomb helyezkedik el: a Saját pozíció, illetve az Útvonaltervezés.
A Saját pozíció gomb használhatához szükség van GPS funkcióra a telefonban.
Ha ez nincs engedélyezve, az alkalmazás egy felugró ablak segítségével átirányítja a felhasználót a GPS funkció bekapcsolására alkalmas képernyőre.
Visszatérve az alkalmazásba, a térképen megjelenik a felhasználó feltételezhető helyzete egy kék ponttal jelölve.
Továbbá a térképen megjelenik egy világoskék kör az előbb említett kék pont körül.
A világosabb színezetű kör területén valamelyik pont a felhasználó biztos pozícióját jelöli, a kör nagysága az internet elérési módjától (mobilnet vagy Wifi) függ.
Ez a funkció azoknak a felhasználóknak nyújt segítséget, akik számára Veszprém városa ismeretlen terület.
%kép: Saját pozíció, dupla mezős útvonaltervezés
A jobb alsó gombra kattintva a felhasználó átkerül az Útvonaltervezés oldalra, ahol a felső beviteli mező mellé egy újabb mező kerül.
Ha az előző oldalon a mezőbe került már földrajzi hely, az alkalmazás automatikusan az alsó mezőt, az Utazás célját tölti fel vele.


\section {Menetrendek}}
\label {menetrendek}

\section {Megállók}
\label {megallok}

\section {Kedvencek}
\label {kedvencek}

%4.2
\section{Admin oldal}
\label{admin}}




