%4. fejezet

A fejezetben bemutatásra kerül a szakdolgozat keretében fejlesztett Android alkalmazás felhasználói szemszögből.
Kifejtésre kerül, milyen menüpontok találhatóak az applikációban, és ezek milyen funkcionalitással rendelkeznek.
Továbbá ismertetem az adatok menedzselésére szolgáló weboldal felületét és működését, amely az admin felhasználók munkáját teszi könnyebbé.

%4.1
\section{Android alkalmazás}
\label{androidapp}
Ahogy \aref{fig_4/icon}.\ ábrán is látszik, az applikáció ikonja egy Veszprém járműparkjában is szereplő autóbusz, az Ikarus 280.
\Picture{Az alkalmazás ikonja}{4/icon}{width=2cm}
Az alkalmazás indítása után az alkalmazás lekéri az adatbázisból azokat az adatokat, amik módosítási dátuma későbbi, mint az utolsó letöltés ideje.
Ezt a felhasználó egy felugró ablak képében látja, amelyen az alkalmazás közli, hogy frissítés van folyamatban, és kéri a felhasználók türelmét.
Ezt \aref{fig_4/fomenu}.\ ábra a) képén tekinthető meg.
Amint az adatok aktualizálása befejeződött, a felugró ablak eltűnik, átadva helyét a főmenü menüpontjainak, melyet \aref{fig_4/fomenu}.\ ábra b) része mutat.
\subsection{Főmenü}
\label{fomenu}
A főmenü az alkalmazás nevét és négy almenüt foglal magába:
\begin{itemize}
	\item Útvonaltervezés
	\item Menetrendek
	\item Megállók
	\item Kedvencek
\end{itemize}

Az applikáció témája a lila szín és annak árnyalatai, amely lehetővé teszi a felhasználók figyelmének felkeltését, mégis letisztult külsőt kölcsönöz.
\Picture{Főmenü}{4/fomenu}{width=10cm}

\subsection {Útvonaltervezés}
\label {utvonalterv}
Az Útvonaltervezés menüpontot kiválasztva egy Google Maps térkép jelenik meg Veszprém városára fókuszálva.
A képernyőn ezen kívül egy beviteli mező és három gomb található.
A beviteli mezőt a felhasználók a térképen való keresésre használhatják.
A keresés fő fókusza Veszprémre irányul, a mező automatikusan próbálja kiegészíteni a begépelt helyszínt, Veszprém környékére összpontosítva, ahogy az \aref{fig_4/autocomplete}.\ ábra a) képén is látható.
Az alkalmazás képes más földrajzi helyre irányuló kereséseket is kiegészíteni, viszont ez a funkció - az alkalmazás minél effektívebb működése érdekében - le van korlátozva.
\Picture{Helység keresése az applikációban}{4/autocomplete}{width=10cm}
Ahogy \aref{fig_4/autocomplete}.\ ábra b) képén is látszik, a találatot az alkalmazás egy úgynevezett 'marker' lerakásával jelöli a térképen.
A bal oldalon található gombbal a felhasználó képes a térkép nézetének megváltoztatására.
Az alapérmezett mód a domborzati megjelenés, a gomb megnyomásával átválthatunk műholdas nézetre.
A jobb oldalon két gomb helyezkedik el: a Saját pozíció, illetve az Útvonaltervezés.
A Saját pozíció gomb használatához szükség van GPS funkcióra a telefonban.
Ha ez nincs engedélyezve, az alkalmazás egy felugró ablak segítségével átirányítja a felhasználót a GPS funkció bekapcsolására alkalmas képernyőre.
\Picture{Útvonaltervezés saját pozícióval}{4/sajatpozUtvonalterv}{width=10cm}
Visszatérve az alkalmazásba, a térképen megjelenik a felhasználó feltételezhető helyzete egy kék ponttal jelölve.
Továbbá a térképen megjelenik egy világoskék kör az előbb említett kék pont körül.
Ezt \aref{fig_4/sajatpozUtvonalterv}.\ ábra a) képén láthatjuk.
A világosabb színezetű kör területén valamelyik pont a felhasználó biztos pozícióját jelöli, a kör nagysága az internet elérési módjától (mobilnet vagy Wifi) függ.
Ez a funkció azoknak a felhasználóknak nyújt segítséget, akik számára Veszprém városa ismeretlen terület.
A jobb alsó gombra kattintva a felhasználó átkerül az Útvonaltervezés oldalra, ahol a felső beviteli mező mellé egy újabb mező kerül, ahogy az \aref{fig_4/sajatpozUtvonalterv}.\ ábra b) részén is látszik. 
Ha az előző oldalon a mezőbe került már földrajzi hely, az alkalmazás automatikusan az alsó mezőt, az Utazás célját tölti fel vele.
Az új képernyőn megjelenő ablakban is lehetőség van a saját pozíció lekérésére, ebben az esetben az alkalmazás a paramétereket a Kiindulási pont mezőbe tölti be.
Abban az esetben, ha mind a kettő mezőt kitöltötte a felhasználó, akkor az Útvonaltervezés gomb ismételt megnyomásával átkerül a találati listára.
Az alkalmazás \aref{fig_4/uttervtalalat}.\ ábra a) részén látható módon jeleníti meg a találatokat.
\Picture{A találati képernyő és az térképes útvonalterv}{4/uttervtalalat}{width=10cm}
A képernyő felső részén a beállított indulási és érkezési helyszín van feltüntetve, alul pedig három féle szempont alapján a megoldás:
\begin{itemize}
	\item Legkevesebb utazási idő
	\\Ennél a lehetőségnél az alkalmazás kiszámolja a lehetséges útvonalakban a gyaloglással és utazással töltött idejét, és ezek közül a legrövidebbet jeleníti meg.
	\item Legkevesebb gyaloglás
	\\A második opció azt a találatot listázza ki, amelynél a gyaloglással töltött idő a legrövidebb.
	\item Legkevesebb átszállás
	\\Az utolsó alternatíva a jeggyel utazóknak kínál utazási megoldást, hiszen a legkevesebb átszállással járó utazást jeleníti meg.
\end{itemize}
A felhasználó az egyik megoldásra kattintva választhatja ki a számára optimális útvonalat.
Az alkalmazás a következő képernyőn az útvonalat térképen jeleníti meg, ahogy az \aref{fig_4/uttervtalalat}.\ ábra b) képén is látszik.
A kezdő és a végpont jelölője piros színű, a busz útvonalát az eltérő színnel (jelen esetben lilával) jelölt megállók mutatják.
Ha az útvonalban több busz is szerepel, akkor azok útvonalát egy harmadik színnel jelöli az alkalmazás.
A gyalogos útvonalat az applikáció színes vonallal jelöli a térképen a piros jelölők és az első/utolsó buszmegállók között.
Ha a választott megoldás mégsem felel meg a felhasználó igényeinek, a képernyő felső részén elhelyezkedő Útvonaltervezés gomb ismételt megnyomásával visszajut az utazási lehetőséget listázó oldalra.
\subsection {Menetrendek}
\label {menetrendek}
\Picture{Menetrendek}{4/menetrend}{width=10cm}
A Menetrendek a főmenü második menüpontja.
Ez az opció azokat a felhasználókat segíti, akik már ismerik Veszprém városát, illetve tömegközlekedését legalább alap szinten.
A menüpontra kattintva az alkalmazás kilistázza a buszjáratok számait, mely \aref{fig_4/menetrend}.\ ábra  a) képén tekinthető meg.
Amint kiválasztott a megtekinteni kívánt járatot, az applikáció bekéri a járat irányát is, majd kilistázza az indulási időket \aref{fig_4/menetrend}.\ ábra b) részén látható módon.
A megnyíló ablak fejléce tartalmazza az előbbi felhasználói akciók során bekért adatokat, illetve a Kedvencekhez adás funkciót egy szív ikon formájában.
A képernyő bal oldalán a járat megállói láthatóak, a választott iránytól függő sorrendben.
Az időpontok jobb oldalon, munkanapokra és szabadnapokra felosztva láthatóak.

\subsection {Megállók}
\label {megallok}
A Megállók menüpont alatt egy Google Maps térkép jelenik meg, rajta Veszprém város buszmegállóival.
A megállók pontos helyét piros színű jelölők jelzik a térképen, amely a közelítés mértékével arányosan csoportosítja a jelölőket.
A felhasználó megkeresi a térképen azt a megállót, amelyikre kíváncsi, az alkalmazás pedig kilistázza az ott megálló járatok számát, ahogy \aref{fig_4/megallokedvenc}.\ ábra a) képén is látszik.
\Picture{Megállók és Kedvencek}{4/megallokedvenc}{width=10cm}

\subsection {Kedvencek}
\label {kedvencek}
A menüpontban a Menetrendek opció alatt kedvencnek jelölt járatokat lehet elérni.
Az alkalmazás kilistázza azokat az adott menetirányhoz tartozó járatokat, amelyeket a felhasználó kedvencnek jelölt \aref{fig_4/megallokedvenc}.\ ábra b) ábráján látható módon.
A funkció egy szív ikon formájában jelenik minden menetrend adatlapján.
Alapértelmezetten a szív üres, rájuk kattintva tudja a felhasználó a Kedvencek menüpont alá rakni, ekkor a szív pirosra színeződik.
A listából eltávolítani hasonló módszerrel lehet, a telt szívre rákattintva kikerül a Kedvencekből.
A funkció gyors elérést biztosít így az adott járatok időpontjaihoz, ezzel meggyorsítva a tájékozódást.

%4.2
\section{Admin oldal}
\label{admin}
Az alkalmazás létrehozásakor szükség volt egy olyan kezelőfelület létrehozására is, amelyen keresztül könnyen kezelhetőek az adatbázisba feltöltött adatok.
Mivel az adatok közérdekűek és sok embert érintenek, fontos szempont volt, hogy csak azok férhessenek hozzá, akiknek van jogosultságuk.
\Picture{Autentikáció}{4/adminSignIn}{width=6cm}
Az oldal eléréséhez a felhasználónak igazolnia kell magát egy felhasználónév-jelszó párossal, ahogy az \aref{fig_4/adminSignIn}.\ ábrán is látszik.
\Aref{fig_4/adminStations}.\ ábrán látható admin oldal jelenik meg a sikeres bejelentkezés után.
Bal oldalon található a menü, amiből a felhasználó kiválaszthatja azt az elemet, amit menedzselni szeretne.
Alapértelmezetten a Járat menüpont jelenik meg, ez az adatbázis architektúrájának alapja.
A másik alapvető entitás a Megállók, amelyek a buszmegállók pontos földrajzi helyzetét tárolják.
\Picture{Adminisztrációs főoldal}{4/adminStations}{width=10cm}
Egy megálló felviteléhez hosszúsági és szélességi koordinátákra van szükség, amelynek felvitelét egy beágyazott Google Maps térkép segíti a weboldalon.
Ezt láthatjuk \aref{fig_4/adminStationAdd}.\ ábrán.
A felhasználó kijelöli a térképen a megálló pontos helyét, aminek a koordinátái megjelennek a térkép mellett és az adminisztrátornak már csak nevet kell hozzárendelnie az új rekordhoz.
Hasonló kezelőfelület található a megállók útvonalakhoz való rendelését kezelő menüpontban.
\Picture{Új megálló hozzáadása}{4/adminStationAdd}{width=10cm}
Egy új útvonal létrehozása esetén az adminisztrátor a Google Maps-en elhelyezkedő jelölők közül kiválasztja a megfelelőt.
Ezek a jelölők az adatbázisban található megállókat jelölik a térképen elhelyezve a könnyebb kezelés érdekében, ahogyan \aref{fig_4/adminLineHasStationAdd}.\ ábrán is látható.
Ha egy útvonal módosul (például egy felújítás miatt), akkor az egy új útvonalként felvezetésre kerül, az alapvető útvonal állapota pedig inaktívvá válik.
Amint az adminisztrátor sikeresen az adatbázishoz adott egy buszjáratot, a megfelelő útvonalhoz és járathoz indulási időket rendel.
Ezeket az időket előre definiált kategóriákba menthetjük el, ilyen kategória például a szabadnap, a munkanap vagy a tanszüneti munkanap.
Mindemellett előfordulhatnak olyan időpontok a naptárban, amikor nem az alapértelmezett közlekedési rend a mérvadó, például hétköznapra eső ünnepnapokon.
\Picture{Új útvonal hozzáadása}{4/adminLineHasStationAdd}{width=10cm}
Mivel ezek évről évre változnak, így az adminisztrátor feladata, hogy megfelelően felvezesse ezeket a napokat az adatbázisba.

Erre szolgál az Eltérő közlekedési rendek menüpont, ahol a megadott időintervallumra vonatkozóan felülírhatjuk az alapértelmezett rendet.

Az adatok átláthatóságának érdekében táblázatban jelennek meg az adatbázisból lekért rekordok.
Az adminisztrátor a keresés funkció segítségével kereshet az adatok attribútumai között, illetve sorba rendezheti azokat a gyorsabb kezelés érdekében.




















