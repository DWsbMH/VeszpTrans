%2. fejezet

A fejezetben ismertetésre kerül az útvonaltervezés, mint szolgáltatás. 
Sok weboldal és telefonos alkalmazás segítségével tudjuk az útunkat előre megtervezni, továbbá különböző egyedi funkciókat is igyekeznek fejleszteni, ezzel csalogatva magukhoz a felhasználókat. 
A fejezet első részében az útvonaltervezést fogom bemutatni pár népszerű alkalmazáson keresztül, amely sikeresen elégíti ki az utazni vágyók igényeit. 
Ezután áttekintést adok Veszprém tömegközlekedéséről, hogy átfogó képet adjak a város közlekedési helyzetéről. 
Végül pedig bemutatom a jelenleg is a piacon lévő segítséget, amik a városban való tájékozódást segítik. 


%2.1
\section{Az útvonaltervezés}
\label{utvonalterv}




\section{Veszprém tömegközlekedése}
\label{veszpremtomeg}



\section{Jelenlegi megoldások}
\label{megoldasok}




\section*{ÉNYKK}
\label{enykk}

Az ÉNYKK vagyis az Észak-nyugat-magyarországi Közlekedési Központ felel a tömegközlekedés üzemeltetése. 
Weboldalukon található dokumentum magába foglalja az összes buszjáratot, és az azokhoz tartozó megállókat és inulási időket. 
Ez a fajta statikus megoldás általános segítséget nyújt az utazni vágyóknak, ha már rendelkeznek információval a városról, például a megállók elhelyezkedést illetően. 
%pdfből kivágás járatról?
A társaság igyekszik több segítséget nyújtani az utasoknak, emiatt új funkciókat készítettek a weboldalra az elmúlt időben. 
Létrehoztak egy térképes funkciót, ahol a járatokat kiválasztva az alkalmazás felrajzolja ezen buszoknak az útvonalát a térképre. 
Továbbá elkezdtek fejleszteni egy útvonaltervező funkciót is, viszont ezek a jelenleg kezdetleges formában működnek csak. 
%térképes kirajzolásról kép



\section*{BamBusz}
\label{bambusz}

Online felületen elérhető segítség, célközönsége főleg az egyetemisták. 
Kedvezőbb megoldást nyújt, mint a busztársaság oldala abból a szempontból, hogy nem kell átböngésznünk az egész dokumentumot az indulási időkért, hanem beállíthatjuk az indulási és az érkezési megállónkat. 
Ezt követően az oldal kilistázza nekünk azokat a buszokat és a hozzájuk tartozó indulási időket, amikkel eljuthatunk a célunkhoz. 
Hátránya hasonlóan a hivatalos oldalhoz, hogy ismernünk kell a megállókat, ahhoz hogy használni tudjuk. 
%bambusz kép

\section*{Veszprémi buszmenetrend}
\label{veszprbuszmen}

Okostelefonra elérhető alkalmazás, ami letisztultan, egyszerűen és gyorsan jeleníti meg a buszjáratokat külön menüpontba szedve. 
Előnye, hogy akár útközben tudunk információt szerezni az autóbuszok közlekedési rendjéről. 
%kép az alkalmazásról??

\newpage







