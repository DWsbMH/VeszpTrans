%2. fejezet

A fejezetben ismertetésre kerül az útvonaltervezés, mint szolgáltatás. 
Sok weboldal és telefonos alkalmazás segítségével tudjuk az utunkat előre megtervezni, továbbá különböző egyedi funkciókat is igyekeznek fejleszteni, ezzel csalogatva magukhoz a felhasználókat. 
A fejezet első részében az útvonaltervezést fogom bemutatni pár népszerű alkalmazáson keresztül, amely sikeresen elégíti ki az utazni vágyók igényeit. 
Ezután áttekintést adok Veszprém tömegközlekedéséről, hogy átfogó képet adjak a város közlekedési helyzetéről. 
Végül pedig bemutatom a jelenleg is a piacon lévő megoldásokat, amik a városban való tájékozódást segítik. 


%2.1
\section{Az útvonaltervezés és a Google Maps}
\label{utvonalterv}

Útvonaltervezőnek nevezzük az olyan szoftvereket, amelyek két földrajzi pont között keresnek optimális útvonalat egy keresőmotor segítségével. 
Ezen motorok feladata, hogy megtalálja az otpimális útvonalat a két pont között, gyakran intermodális működésűek. 
Már az 1970-es évektől használják a támogatás ezen fajtáját. 
Akkoriban ez annyit jelentett, hogy egy terminálos felhasználói interfészen keresztül csatlakozott a hívóközponthoz, és onnan érdeklődték meg a tömegközlekedéssel kapcsolatos információkat. 
Miután elterjedt az a szokás az emberek között, hogy maguknak tervezték meg a nyaralásokat, és nem vették igénybe az utazási irodák ügynökeit, elkezdtek fejlődni az interneten elérhető útvonaltervezők. 

A tervezők ebbe a fajtájába tartozik az akkoriban Google Transitként ismert útvonaltervező, ami napjainkban a Google Maps térképes szolgáltatás része. 
Az alkalmazás interneten és mobil eszközökre is elérhető, rengeteg plusz funkcióval. 
A térképes adatbázisát különböző partnerek segítségével szerzi be, de sok helyen (például a fejlődő országokban) a közösség frissíti a térképes adatokat. 
Az útvonaltervező funkció kezdetben gyalogos és autós közlekedés tervezésére volt képes, azonban 2007-ben integrálták a tömegközlekedést is az útvonaltervezésbe. 
Magyarországon 2011 óta kizárólag Budapesten érhető el a funkció. 

\Picture{Google Maps útvonaltervezés}{2/googlemaps}

Ahogy a \ref{fig_2/googlemaps}. ábrán is látszik, tervezéskor beállíthatjuk a közlekedési formát, hogy éppen gyalogosan vagy tömegközlekedéssel szeretnénk igénybe venni. 
Gyalogos és autós közlekedéskor az elérhető járdákat és autóutakat veszi figyelembe az alkalmazás, majd az eredményt kirajzolja a térképre, esetlegesen több elérhető opció esetén a többit szaggatott vonallal jelöli. 
Tömegközlekedés esetén is több lehetőséget kínál fel, ezekből több szempont alapján tudjuk kiválasztani a számunkra optimálisat. 
Ilyen szempont lehet a legkevesebb átszállállás, illetve legkevesebb gyaloglás. 
A Google Maps továbbá indulási időket is rendel a járatokhoz, így képesek vagyunk tervezni mostani időpillanathoz illetve ha később szeretnénk csak utazni azt is beállíthatjuk. 
Továbbá rendelkezik élő menetrendinformációval, amit a Budapesti Közlekedési Központ hivatalos oldaláról szerez be. 
Itt láthatjuk ha felújítás, útlezárás miatt nem közlekednek járatok, vagy más útvonalon járnak emiatt más megállókat érintenek, ezt láthatjuk a \ref{fig_2/elojaratinfo}. ábrán is. 

%túl nagy
\Picture{Élő útvonalinformáció}{2/elojaratinfo}{width=10cm}


\section{Veszprém tömegközlekedése}
\label{veszpremtomeg}

Veszprém tömegközlekedését jelenleg a város méretéből és rendelkező álló infrastruktúrájából adódóan autóbuszok adják. 
A buszokat a Balaton Volán Zrt. biztosítja az 1960-as évek óta. 
Veszprém tömegközlekedésének gondolata azonban már 1884-ben megfogalmazódott Czollenstein Ferenc által, aki omnibuszokat indított Veszprém-Balatonalmádi között. 
A jelenlegi közlekedési forma 28 vonalat foglal magában, és a város belterületén kívül közlekedik a közigazgatásilag a városhoz tartozó településekhez is, úgymint  Szabadságpuszta, Jutaspuszta, Kádárta és Gyulafirátót, valamint Csatár. 
Több átalakításon is átesett, amíg elérte a mai napi formáját, amit a \ref{fig_2/veszprembuszhalozat}. ábra mutat.
\Picture{Veszprém tömegközlekedési hálózata}{2/veszprembuszhalozat}{width=10cm}

A menetrend integrálása lehetséges a Google Maps rendszerébe, így az útvonaltervezés funkció Veszprém városában is használható lehetne. 
A busztársaság részéről, egy meghatározott formátumú adatbázis továbbítása szükséges a Google felé, mivel ez csak Budapesten valósult meg, így csak a fővárosban érhető el Magyarországon belül ez a szolgáltatás. 





\section{Jelenlegi megoldások}
\label{megoldasok}

Veszprémben a tömegközlekedés támogatására jelenleg is létezik több megoldás. 
A busztársaság is igyekszik minél kielégítőbb segítséget nyújtani az utazóközönségének, hiszen fontos számára, hogy minél többen vegyék igénybe a tömegközlekedést. 
Továbbá léteznek olyan harmadik fél által készült eszközök is, amelyek szintén hozzájárulnak az információszerzéshez. 
Ezen alkalmazásokat fejlesztőik nem profitszerzési céllal készítették el, hanem csupán önkéntes alapon, az emberek megsegítésének céljával, ezáltal funkcionalitásuk elmarad egy céges környezetben, nagyobb fejlesztőgárda által készített szolgáltatástól, melytől nyereséget várnak a tulajdonosok. 
Ezek közül mutatok be pár ismertebb példát amik jelenleg elérhetőek a piacon. 



\section*{ÉNYKK}
\label{enykk}

Az ÉNYKK vagyis az Észak-nyugat-magyarországi Közlekedési Központ felel a tömegközlekedés üzemeltetéséért. 
Weboldalukon található dokumentum magába foglalja az összes buszjáratot, és az azokhoz tartozó megállókat és indulási időket. 
A \ref{fig_2/menetrend}. ábrán látható módon szerepel egy buszjárat a dokumentumban. 
Ez a fajta statikus megoldás általános segítséget nyújt az utazni vágyóknak, ha már rendelkeznek információval a városról, például a megállók elhelyezkedését illetően. 

%túl nagy
\Picture{Az 1-es buszhoz tartozó menetrend táblázat}{2/menetrend}{width=10cm}

A társaság igyekszik több segítséget nyújtani az utasoknak, emiatt új funkciókat készítettek a weboldalra az elmúlt időben. 
Létrehoztak egy térképes funkciót, ahol a járatokat kiválasztva az alkalmazás felrajzolja ezen buszoknak az útvonalát a térképre. 
Továbbá elkezdtek fejleszteni egy útvonaltervező funkciót is, viszont ezek jelenleg kezdetleges formában működnek csak. 
A térképes szolgáltatásnál kiválaszthajtuk, hogy milyen buszjáratokra vagyunk kiváncsiak, és a program kirajzolja azokat a térképre, ahogy a \ref{fig_2/kirajzoltjarat}. ábra mutatja. 

\Picture{Az 1-es és a 4-es járat útvonala}{2/kirajzoltjarat}{width=10cm}


\section*{BamBusz}
\label{bambusz}

Online felületen elérhető segítség, célközönsége főleg az egyetemisták. 
Kedvezőbb megoldást nyújt, mint a busztársaság oldala abból a szempontból, hogy nem kell átböngésznünk az egész dokumentumot az indulási időkért, hanem beállíthatjuk az indulási és az érkezési megállónkat. 
Ezt követően az oldal kilistázza nekünk azokat a buszokat és a hozzájuk tartozó indulási időket, amikkel eljuthatunk a célunkhoz a \ref{fig_2/bambusz}. ábrán látható módon. 
Hátránya hasonlóan a hivatalos oldalhoz, hogy ismernünk kell a megállókat, ahhoz hogy használni tudjuk. 

\Picture{A BamBusz webes felülete}{2/bambusz}{width=10cm}

\section*{Veszprémi buszmenetrend}
\label{veszprbuszmen}

Okostelefonra elérhető alkalmazás, ami letisztultan, egyszerűen és gyorsan jeleníti meg a buszjáratokat külön menüpontba szedve, ahogy a  \ref{fig_2/veszpremappmenu}. ábrán látható. 
Előnye, hogy akár útközben tudunk információt szerezni az autóbuszok közlekedési rendjéről. 
Továbbá elérhető egy éves menetrendi naptár a főoldalon, ami segítségével megállapíthatjuk hogy milyen rend szerint közlekednek a buszok adott napokon. 
Ugyanakkor az alkalmazás nem naprakész, a naptár a tavalyi évet reprezentálja, illetve ebből kifolyólag a buszjáratok menetrendje is az elmúlt évre érvényes indulásokat mutatja.

%ezeket egymás mellé, még nem néztem utána hogy kell megoldani
\Picture{Az alkalmazás menüje}{2/veszpremappmenu}{width=10cm}
\Picture{Az alkalmazás főoldala a menetrendi naptárral}{2/veszpremapp}{width=10cm}

\newpage







