%6. fejezet

Az elkészült alkalmazás a felé támasztott követelményeknek eleget tesz, ennek ellenére funkcionalitása tovább bővíthető.
Fontos szempont, hogy a fejlesztés végeztével tesztelési céllal is ellenőrzésre kerüljön az applikáció.
Teszteléskor olyan hibák is jelentkezhetnek, amelyek a felhasználók által használt akciókból erednek.
Ennél a szakasznál lényeges lehet egy külső személy bevonása az ellenőrzésbe, ezáltal lecsökkentve az alkalmazás nem megfelelő használatából eredő hibák számát.
Mindemellett fontos, hogy az applikáció minél magasabb szinten elégítse ki a felhasználói igényeket, így továbbfejlesztésre és új funkciók bevezetésére is szükség lehet.

\section*{Android Room framework}
\label{androidroom}
Az alkalmazás egyszerűsége és fenntarthatóságának érdekében érdemes lenne az előző fejezetben ismertetett Room adatbázis absztrakciós könyvtárat használni.
Segítségével egyszerűbbé és átláthatóbbá válna a kódbázis.

\section*{További városok hozzáadása}
\label{morecity}
Veszprém megyében több kisváros is rendelkezik helyi tömegközlekedéssel, ami egy-kettő járattól akár húsz járatot is jelenthet.
Ebből kifolyólag a továbbiakban fontos lehet a környező városok tömegközlekedésének integrálása.
Ehhez az adatbázis átalakítása, továbbfejlesztése szükséges, hogy az általános érvényű legyen minden városra.
Ez a funkció segítséget nyújtana a kisebb városokban élőknek, illetve azoknak a turistáknak, akik a megye több városát felkeresve nyaralnak.

\section*{Többnyelvűség}
\label{internationalization}
Külföldről érkező turisták esetében megfontolandó a többnyelvűség támogatása.
Ebben az esetben lényeges lehet egy nyelvválasztó funkció implementálása az alkalmazásba.
Első sorban az angol és német nyelvek megvalósítása, de a későbbiekben akár még több nyelv bevonásával.

\section*{Útvonaltervezés továbbfejlesztése}
\label{routeplan}
Az alkalmazás optimálisabb működése érdekében fontos tényező lehet az útvonaltervezés továbbfejlesztése.
A felhasználói élmény növekedése érdekében a tervező algoritmus pontosabbá, gyorsabbá tétele.
Ehhez a jelenlegi megoldást le lehetne cserélni okosabb, heurisztikus algoritmusra.
Az útvonaltervező mellett a felhasználói felületet is tovább lehet fejleszteni, ilyen feladat lehet többek közt a járatok megjelenítésének pontosítása. 

\section*{Mobil értesítések}
\label{pushnot}
További ötletként felmerült, hogy a felhasználók számára értesítéseket küldjön ki az alkalmazás.
A felhasználó beállíthatná a tartózkodási helyét, illetve egy járat indulási időpontját, az alkalmazás pedig jelezné egy értesítés formájában, ha a felhasználónak indulnia kell a megadott helyről, hogy elérje a korábban beállított járatot.
Olyan esetben is hasznos lenne ez a funkció, ha útvonalváltozás történik egy buszjáratnál, így a felhasználó időben értesítve lenne a módosításról.

\section*{Widget}
\label{widget}
Érdemes lenne megvalósítani egy olyan modult, amely kitehető a telefon főképernyőjére.
Az alkalmazásban, a Kedvencek menüpont alatt beállított járatok táblázatai között válthatna a felhasználó, így az alkalmazás elindítása nélkül, hamar megtekinthető lenne az egyes járatok menetrendje.
Mindemellett fontos, hogy ez a modul is folyamatosan friss információkat jelenítsen meg, ha elérhető frissebb adat, az legyen jelezve a felhasználó felé is.
\newpage