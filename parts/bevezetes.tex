%1. fejezet

A nagyvárosokban található tömegközlekedés bonyolult és szerteágazó módon húzódik végig a város különböző pontjait érintve. 
A tömegközlekedéshez tartozó útvonalak, megállók és indulási idők sokaságán az tud igazán kiigazodni, aki hosszabb ideje használja már. 
A városba „idegenként” érkezők, turisták számára szükséges lehet egy olyan eszköz, melynek segítségével eljutnak a céljukhoz, gyorsan és kényelmesen tudják használni a város nyújtotta közösségi közlekedést. 

Veszprémben jelenleg a közlekedés e fajtáját az autóbuszok szolgálják ki. 
Bár a megyeszékhely nem tartozik a legnagyobb városok közé Magyarországon, mégis szerteágazó menetrendet tudhat magáénak. 
Emellett a városban gyakran fordulnak meg egész évben turisták, egyetemisták, akik számára a legnagyobb hátrány, hogy nem ismerik a buszmegállókat, esetlegesen a saját helyzetüket sem. 
Egyelőre még nem található olyan szolgáltatás városunkban, ami eleget tenne annak, hogy segítse az utazóközönséget a tájékozódásban.
Jelenleg több, a veszprémi tömegközlekedést segítő megoldással is találkozhatunk. 
Ezen megvalósítások ugyanakkor túlságosan is statikusak egy újonnan a városba látogató számára. 
Egyik ilyen fennálló probléma az, hogy megtudja az utas egy adott megállóban milyen buszok állnak meg, végig kell néznie az egész menetrendet, továbbá nem szolgálnak vizuális visszajelzéssel, azaz ugyan ismeri a megálló nevét, azonban nem tudja pontosan meghatározni a város mely területén található. 
Ilyen helyzetekben igény lenne egy olyan megoldásra, hogy térkép alapján is tudjanak tájékozódni, viszont a jelenlegi megoldások közül egyik sem felel meg az előzőekben felállított igényeknek a kielégítésére.
Erre a fennálló problémára nyújt megoldást a szakdolgozatomban megvalósított Android alkalmazás, amely hordozható, megbízható és gyors formában tájékoztatja az utazni kívánókat. % Do or don't there is no try

A 2. fejezetben bemutatom az útvonaltervezést, mint szolgáltatást, illetve a városban működő jelenlegi megoldásokat. 
A 3. fejezetben ismertetem a program szükséges funkcióit, és az azokkal szemben támasztott követelményeket.
A 4. fejezetben bemutatom az elkészült alkalmazást felhasználói szinten.
Az 5. fejezetben kitérek a fejlesztésre részletesebben, az alkalmazás egyes részegségeire és azok implementációjára. 
Végül a 6. fejezetben felvázolok pár elképzelést az alkalmazás továbbfejlesztéséhez.
\newpage
